%%%%%%%%%%%%%%%%%%%%%%%%%%%%%%%%%%%%%%%%%%%%%%%%%%%%%%%%%%%%%%%%%%%%%%%%%%%%%
%%%%%%                                                                  %%%%% 
%%%%%%          Maqueta de memòria TFC/PFC de l'EETAC                   %%%%% 
%%%%%%                                                                  %%%%% 
%%%%%%%%%%%%%%%%%%%%%%%%%%%%%%%%%%%%%%%%%%%%%%%%%%%%%%%%%%%%%%%%%%%%%%%%%%%%%
%%%%%%%%%%%%%%%%%%%%%%%%%%%%%%%%%%%%%%%%%%%%%%%%%%%%%%%%%%%%%%%%%%%%%%%%%%%%%
%%                                                                         %%
%%          Autor: Xavier Prats i Menéndez (xavier.prats@upc.edu)          %% 
%%                  Technical University of Catalonia (UPC)                %%
%%                                                                         %%
%%%%%%%%%%%%%%%%%%%%%%%%%%%%%%%%%%%%%%%%%%%%%%%%%%%%%%%%%%%%%%%%%%%%%%%%%%%%%
%%      This work is licensed under the Creative Commons  Attribution-     %%
%%   -Noncommercial-Share Alike 3.0 Spain License. To view a copy of this  %% 
%%    license, visit http://creativecommons.org/licenses/by-nc-sa/3.0/es/  %%
%%    or send a letter to Creative Commons, 171 Second Street, Suite 300,  %%
%%                  San Francisco,California, 94105, USA.                  %%
%%%%%%%%%%%%%%%%%%%%%%%%%%%%%%%%%%%%%%%%%%%%%%%%%%%%%%%%%%%%%%%%%%%%%%%%%%%%%
%% Versió 2.1 - Juliol 2012                                                %%
%%%%%%%%%%%%%%%%%%%%%%%%%%%%%%%%%%%%%%%%%%%%%%%%%%%%%%%%%%%%%%%%%%%%%%%%%%%%%

%%% NOTA: els seguents packages son necessaris per utilitzar la
%%%       plantilla seguent:
%%%       ifthen,calc,helvet,pslatex,fancyhdr,nextpage,subfigure,tocloft,graphicx,url

%%% NOTA: Es possible que algunes distribuicions Linux o Windows.
%%%       no portin aquests paquets instal·lats per defecte.
%%%       En aquest cas els haureu d'instal·lar manualment.


%%%%%%%%%%%%%%%%%%%%%%%%%%%%%%%%%%%%%%%%%%%%%%%%%%%%%%%%%%%%%%%%%%%%%%%%%%%%%
% 1- INICIALITZACIÓ
%%%%%%%%%%%%%%%%%%%%%%%%%%%%%%%%%%%%%%%%%%%%%%%%%%%%%%%%%%%%%%%%%%%%%%%%%%%%%

\documentclass[catalan,final]{setup/eetac_tfc_pfc}
%% * OPCIONS A CONFIGURAR al \documentclass
%%    - Estat del document: final o draft
%%      NOTA: Draft no inserta les figures i marca només l'espai que
%%      ocupen. També s'indica quan el text sobrepassa els marges.
%%      Draft és molt útil per compilar ràpid el document si no és important
%%      en aquell moment visualitzar les figures.
%%    - Idioma PRINCIPAL del document: catalan, spanish, english, french...

\usepackage[english,catalan]{babel}
%%  * INCLOURE TOTS ELS IDIOMES QUE S'USARAN EN EL DOCUMENT
%%    NOTA: per canviar d'idioma al mig del document usar:
%%          \selectlanguage{nom_idioma}
%%%%%%%%%%%%%%%%%%%%%%%%%%%%%%%%%%%%%%%%%%%%%%%%%%%%%%%%%%%%%%%%%%%%%%%%%%%%%

%%%%%%%%%%%%%%%%%%%%%%%%%%%%%%%%%%%%%%%%%%%%%%%%%%%%%%%%%%%%%%%%%%%%%%%%%%%%%
% 2- CÀRREGA DE PAQUETS ADICIONALS (OPCIONALS)
%%%%%%%%%%%%%%%%%%%%%%%%%%%%%%%%%%%%%%%%%%%%%%%%%%%%%%%%%%%%%%%%%%%%%%%%%%%%%

%%% NOTA: Es possible que algunes distribuicions Linux o Windows.
%%%       no portin aquests paquets instal·lats per defecte.
%%%       En aquest cas els haureu d'instal·lar manualment.

%% El paquet inputenc és extramadament útil. 
%% Permet escriure els accents directament amb l'editor de texte
%% sense haver de fer coses com per exemple: introducci\'o
%% Heu d'especificar la codificació de caracters que utilitzeu pel
%% vostre fitxer (en aquest exemple utf8)
\usepackage[utf8]{inputenc}

%% Símbols matemàtics de la American Mathematical Society
\usepackage{amssymb,amsmath, amsfonts}  

%% El paquet array proporciona eines molt útils a l'hora de fer 
%% equacions amb matrius
\usepackage{array}             

%% Paquet que permet fer taules fusionant cel·les de files consecutives
\usepackage{multirow}          

%% Paquet molt útil en cas de tenir taules molt llargues que 
%   ocupin vàries pàgines
\usepackage{longtable}          

%% Permet canviar els colors del document
%\usepackage{color,colortbl}

%% Paquet molt útil que permet activar links en el PDF final.
%% * NO OBLIDAR DE CONFIGURAR els quatre primer camps!
\usepackage[
  pdfauthor={Nom Cognoms autor},            % Configurar adientment
  pdftitle={Treball Fi de Grau - Sergi Garreta Serra}, % Configurar adientment
  pdfsubject={Titol del TFG aqui},          % Configurar adientment  
  pdfkeywords={keyword1, keyword2, ...},    % Configurar adientment
  pdfcreator={EETAC-UPC}, 
  pdfproducer={LaTeX, dvipdf},
  pdfdisplaydoctitle=true, plainpages=false, linktocpage=true,         
  colorlinks=true, linkcolor=blue,citecolor=blue,urlcolor=blue,
  hyperfootnotes=false, pagebackref=true, pdfpagelabels=true,
  pdfpagemode=UseOutlines,
]{hyperref} 

%% NOTA IMPORTANT!:
%% Per tal que hyperef funcioni correctament amb els capitols o seccions no
%% numerats (\chapter*{}), com per exemple introducció, conclusions i bibliografia
%% cal posar les dues comandes seguents ABANS del \chapter*{} en questió
%\cleardoublepage
%\phantomsection

%% Permet trencar links URL. 
%% Atenció! afegir aquest paquet DESPRES del hyperref!!
\usepackage{breakurl} 

%% Permet arranjar matricialment multiples figures
%% NOTA: afegir aquest paquet DESPRES del hyperref!!
%%       Si no es desitja utilitzar aquest paquet, comentar la linia seguent
%%       i anar TAMBE al fitxer de classe (eetac_tfc_pfc.cls) per substituir: 
%%       \RequirePackage[subfigure]{tocloft}  per  \RequirePackage{tocloft}
\usepackage{subfigmat}         

\usepackage{csvsimple}
\usepackage{rotating}

%%%%%%%%%%%%%%%%%%%%%%%%%%%%%%%%%%%%%%%%%%%%%%%%%%%%%%%%%%%%%%%%%%%%%%%%%%%%%


%%%%%%%%%%%%%%%%%%%%%%%%%%%%%%%%%%%%%%%%%%%%%%%%%%%%%%%%%%%%%%%%%%%%%%%%%%%%%
% 3- DOCUMENT
%%%%%%%%%%%%%%%%%%%%%%%%%%%%%%%%%%%%%%%%%%%%%%%%%%%%%%%%%%%%%%%%%%%%%%%%%%%%%

%%% Configuració de les dades i variables boleanes rellevants del document:
\input{setup/dades.tex}  

%%% Configuració de MACROS o ENTORNS (opcionals) definides per l'usuari:
\input{setup/user-macros.tex}  

%%% Configuració manual de les regles d'hyphenation:
\input{setup/hyphenation.tex}  

\begin{document}

%% Seleccionar l'idioma principal del document:
\selectlanguage{catalan}

\beforepreface

%% RESUM i OVERVIEW
%%%%%%%%%%%%%%%%%%%%%%%%%%%%%%%%%%%%%%%%%%%%%%%%%%%%%%%%%%%%%%%%%%%%%%%%%%%%%
% NOTA: les longituds passades com a parametres d'entrada  s'han
%        d'ajustar manualment fins que el requadre del resum/overview
%        ocupi tota la pàgina. 

%%% Resum en català (o castellà)
\selectlanguage{catalan}   
\begin{resum}{10cm}
Avui en dia tots ells telefons intel·ligents tenen la tecnologia Bluetooth però molt poca gent sap realment que Bluetooth té dues version diferents.
En aquest treball es tractarà el Bluetooth Low Energy que s'utilitza per transmetre informació rapidament i amb el menor consum d'energia possible.
\newline
\newline
L'objectiu d'aquest treball és entendre fàcilment com funciona per dins BLE, en que es diferencia respecte el Bluetooth Clàssic i quins avantatges aporta en xarxes de sensors.
Posteriorment s'analitzaran exemples reals d'implementacions per entendre en quins casos aquesta tecnologia és més útil.
\newline
\newline
Es disenyaran i desenvoluparan escenaris, utilitzant el xip CC1352R de Texas Instruments, utilitzant BLE per a la comunicació de sensors que prenguin mesures mediambientals i biològiques.
Es realitzarà un estudi de quin és el consum real d'els dispositius per veure quin temps de vida poden arribar a tenir.
També s'analitzaran altres mètriques com la distància màxima i latència de la comunicació.
\newline
\newline
S'explicarà pas a pas com s'ha desenvolupat per tal de poder utilitzar aquest treball com a base per a poder fer o modificar implmentacions similars.
\end{resum}

%%% Resum en anglès
\selectlanguage{english}   
\begin{overview}{11cm}
Overview
\end{overview}

% Tornar a l'idioma principal del document
\selectlanguage{catalan}  

%NOTA: En cas d'utilitzar l'espanyol com a idioma principal del document, el
%      latex anomena les taules com a 'Cuadros'. Si es desitja canviar aquesta
%      nomenclatura i utilitzar la paraula 'Tabla' descomentar les línies següents:
%\def\listtablename{Índice de tablas}
%\def\tablename{Tabla}%



% Amb aqueta comanda indiquem que ja s'han inclòs tots els apartats del prefaci del 
% projecte o podem començar a incloure els capitols de la memòria
\afterpreface


%%%%%%%%%%%%%%%%%%%%%%%%%%%%%%%%%%%%%%%%%%%%%%%%%%%%%%%%%%%%%%%%%%%%%%%%%%
%%%%%% INCLOURE A PARTIR D'AQUÍ TOTS ELS CAPÍTOLS DE LA MEMORIA   %%%%%%%%
%%%%%%%%%%%%%%%%%%%%%%%%%%%%%%%%%%%%%%%%%%%%%%%%%%%%%%%%%%%%%%%%%%%%%%%%%%

% NOTA: recordar que la introducció i les conclusions són capítols NO
%       enumerats, per tant s'ha d'usar \chapter*

% NOTA: és aconsellable incloure els capítols de la memòria en fitxers 
%       separats utlitzant la comanda \input  Per exemple:
%       \input{capitol1}  
%       que farà que s'inclogui el fitxer capitol1.tex

% NOTA: Si es vol incloure agraïments i/o glosari, fer-ho utilitzant 
% \chapter*{} i incloure'ls abans la introducció

\cleardoublepage
\phantomsection
\chapter*{Introducció}
La arquitectura internet de les coses que s'ha popularitzat en els últims anys a comportat unes noves necessitats en molts aspectes de la tecnologia que són molt diferents als anteriors.
Quan el que es vol és que tot estigui connectat entre si, queda clar que es trenquen moltes premisses amb les que s'havien dissenyat originalment les tecnologies més importants.

La transició cap a l'Internet Of Things, transforma el paradigma de les comunicacions de pocs elements a molta velocitat (xarxes centralitzades) cap a molts elements amb poca velocitat (xarxes distribuïdes).
Això afecta les tecnologies sense fils que no havien estat dissenyades pel seu ús tant generalitzat.

Aquests canvis es poden veure en l'evolució de la majoria de tecnologies sense fils com WiFi, xarxes cel·lulars i Bluetooth entre d'altres.
En totes aquestes tecnologies ha estat necessari fer grans canvis per acomodar a molts més usuaris dels que s'havien predit inicialment.

El Bluetooth clàssic és un exemple de una tecnologia que tenia limitacions a l'hora de utilitzarse en certs escenaris.
No permetia comunicació entre més d'un dispositiu alhora, tenia un abast limitat, poques mesures per contrarestar les interferències...
Al llarg de la seva història es van definir noves versions que anaven millorant les mancances del protocol.
Tot i així, el cost que suposa haver d'establir una connexió per transmetre dades, encara que siguin molt poques, segueix sent molt alt.

És per això que Bluetooth va incorporar la extensió Low Energy de forma opcional.
En aquesta extensió esta definit el protcol anomenat Bluetooth Low Energy que soluciona el problema de Bluetooth Clàssic ja que permet molt fàcilment i utilitzant molt poca energia transmetre dades (que inclou a múltiples dispositius) sense haver d'establir una connexió.

Però Bluetooth Low Energy també va anar millorant amb les noves versions, millorant el rendiment en escenaris concrets.
Aquestes millores inclouen augmentar el màxim de dades que es poden transmetre sense connexió o augmentar la tassa de dades de transmissió per poder estalviar energia.
En aquest treball es veurà com funciona el BLE, com ha evolucionat al llarg del temps i es farà una implementació real del protocol.
\chapter{Origens del Bluetooth Low Energy}\label{C:compaginacio}

%\section{Ús lliure de RF}
%Amb l'establiment de les bandes ISM (Afegir referencia) es crea la possibilitat per el públic general utilitzar comunicació inal·làmbrica amb molta facilitat. Això es deu a que l'establiment de les bandes ISM està acceptat de manera (mes o menys(?)) de la mateixa manera internacionalment.
%Això facilita el disseny de productes per al consumidor habitual que ràpidament veu els avantatges de la comunicació inal·làmbrica entre dispositius.
%En aquest entorn sorgeix la necessitat d'establir estandards entre le companyies principals dels sectors. Per definir els estandards s'agrupen les companyies i formen grups com al WIFI Alliance o ...

%\subsection{Wifi}
%La tecnologia de comunicació més popular avui en dia és la WIFI, dissenyada originalment als anys (?) i establerta als anys (?) orientada a permetre una connexió (* bona) i sense consideració per les interferències entre si mateixa que no es podien preveure llavors.  
%- Avantatges
%- Innovació
%- Evolució

\section{Història de Bluetooth Clàssic}
El desenvolupament de tecnologia per a connexions de curt abast que acabarien esdevenint el que avui es coneix com a Bluetooth va començar al 1989 per part de Ericsson.
Inicialment la voluntat era utilitzar aquesta tecnologia per a auriculars inal·làmbrics. La tecnologia va ser primer utilitzada en els ordinadors portàtils de IBM (ThinkPad). Inicialment es volia integrar connectivitat de la xarxa de telefonia als ordinadors però no era factible degut al consum d'energia de la tecnologia mòbil.
IBM i Ericsson van acordar utilitzar aquesta tecnologia de curt abast per permetre a mòbils Ericsson i ordinadors ThinkPad comunicar-se entre si i poder accedir a la xarxa mòbil des d'un ordinador.
Com que ni Ericsson ni IBM tenien majoria en la quota de mercat dels respectius productes van decidir que la tecnologia fos oberta. Es van unir al grup Intel, Nokia i Toshiba, i totes 5 companyies el 1998 van fundar el \textit{Blueooth Special Interest group}.
El 2001 van sortir a la venta el primer mòbil amb Bluetooth, el Ericsson T39 i el primer portàtil el IBM ThinkPad A30

\section{Història de Bluetooth Low Energy}
El 2001 a Nokia es va començar a buscar una versió de Bluetooth que fos similar però que reduís significativament el consum d'energia amb el mínim de compromisos possibles.
El 2004 es va publicar la \textit{Bluetooth Low End Extension} \cite{Original_BLE_Extension}. 
Després de més desenvolupament juntament amb Logitech i MIMOSA [citation needed] es va divulgar al 2006 amb el nom de Wibree.

\begin{figure}[h]
	\begin{center}
		\includegraphics[width=0.6\textwidth]{./images/Wibree_Logo.png}
		\caption{Wibree Logo}
	\end{center}
\end{figure}

Després de negociar amb els membres del SIG es va acordar incloure Wibree al estandard de bluetooth en la especificació 4.0 amb el nom de \textit{Bluetooth ultra low power technology} i publicitat com a Bluetooth Smart. El primer mòbil a incloure'l va ser el iPhone 4S al 2011.

\section{Bluetooth vs BLE}
Bluetooth Low Energy no té cap relació amb Bluetooth Classic pel que fa a l'arquitectura del protocol. Tot i compartir nom (perque?) queda clar que no estem parlant del mateix amb el simple fet que no son compatibles entre sí. Tot i així comparteixen l'ús de la banda freqüencial de 2,4 GHz igual que altres protocols sense fils. Al 2016 es va anunciar la versió 5.0 anomenada Bluetooth 5 (sense el punt)[cita] i se li va canviar el nom (aclariment).


\section{MANETS}
Bluetooth clàssic no permetia tenir més d'una connexió establerta amb dispositius i l'arquitectura feia que s'utilitzés bastanta energia per establir i mantenir una connexió encara que es volguessin transmetre molt poques dades.
Això no era un problema inicialment ja que principalment l'ús era per la transferència de fitxers com contactes o fotografies.
Junt amb l'arribada dels telèfons intel·ligents els auriculars inal·làmbrics es van fer populars i també es va establir la dominància de Bluetooth per a escoltar música sense cables.
Amb aquest objectiu es va millorar Bluetooth per tal de ser més resistent a interferències i aconseguir velocitats superiors per poder transmetre música d'alta fidelitat.
Per un altre costat però va començar a sorgir el Internet of Things, basat en tenir molts dispositius connectats transmetent a taxes molt variades i de forma discontinua i això va generar necessitats que no es podien cobrir amb les tecnologies existents.
Era necessari tenir xarxes sense fils, amb consum molt baix d'energia i de llarg abast.

Amb aquests requeriments van sorgir les MANETS (Mobile Ad-Hoc Networks) i a continuació compararem les característiques de les tecnologies més importants amb BLE per veure on queda situat BLE.

\subsection{Versions de BLE}
En quant al estàndard Bluetooth Low Energy cal diferenciar les característiques de la versió 4.0 amb la 5 ja que hi ha canvis significatius \cite{BLE_5_improvement_over_4}.

\subsubsection{LE 2M}
En la versió de Bluetooth 5 permet a la capa física operar a 2 Msimbols/s enlloc dels 1 Ms/s que es podia anteriorment. L'augment de la velocitat dels símbols permet augmentar la quantitat de dades que es poden transmetre a nivell d'aplicació.
Aquest canvi està pensat per facilitar la utilització de BLE en situacions més exigents pel que fa a dades com per exemple múltiples mesures del cos humà.

\subsubsection{Abast}
Tot i que la tecnologia BLE ja de per si en la seva versió original (4.0) tenia una distància màxima de transmissió teòrica molt gran.
Però tenir la possibilitat de més abast en les comunicacions ajuden a cobrir les necessitats generades per sectors com el de les cases intel·ligents.
El límit del abast ve definit per la potència màxima de transmissió (sigui per protocol o per normativa) i la \textit{Bit Error Ratio} que s'accepta sent el percentatge de bits que es reben incorrectament.
Bluetooth defineix la BER mínima en 0.1\% per tant quant un de cada mil bits es erroni es considera que la connexió és massa dolenta i que ja s'ha arribat al límit de distància.
Per tractar els errors en transmissió hi ha dues estratègies: detecció d'errors i correcció d'errors.
En la versió 4.0 de Bluetooth només s'utilitzava la detecció d'errors.
En Bluetooth 5 s'implementa correcció d'errors d'aquesta manera sense augmentar la potència transmesa es pot incrementar el abast màxim que pot tenir la comunicació.
El desavantatge és que més bits dels que es transmeten es dediquen a corregir errors per tant queda menys espai per a dades així reduint el \textit{throughput}.


El esquema de correcció d'errors (\textit{Forward Error Correction}) es fa amb un codificador convolucional que pot generar a la sortida més bits dels que entren. El codificador pot funcionar amb dos esquemes diferents s'anomenen S=2 i S=8.
Amb els diferents esquemes s'utilitza un \textit{code rate} diferent que pot augmentar la capacitat de correcció i augmentar el abast però disminuir la quantitat de dades que es poden transmetre.

\begin{figure}[h!]
	\begin{center}
		\includegraphics[width=0.8\textwidth]{./images/LE_PHY.png}
		\caption{Comparació de diferents capes físiques}
	\end{center}
\end{figure}

\subsubsection{Advertisement Extensions}
\subsubsection{Slot Availability Masks}
\subsubsection{Improved Frequency Hopping}


\subsection{Altres Protocols}
Bluetooth Low Energy no és l'únic protocol orientat a la connexió de dispositius amb baix consum energètic. També existeixen els següents protocols:
Zigbee, utilitzat en Philips Hue
Zwave Ring (home security)
6lowpan
insteon
lorawan

% Prodcutes que utilitzen cadascun

\subsection{Comparació}
Tot i que els protocols tenen molt en comú cada un es diferencia de la resta en els detalls, a continuació veiem una comparació de les capacitats de cada un dels protocols

\section{BLE Stack}
Un cop vistes les característiques generals del protocol cal entendre com funciona per dins.
En entrar en detall queda clar com el protocol s'ha dissenyat de forma molt flexible 

\begin{figure}[h!]
	\begin{center}
		\includegraphics[width=0.6\textwidth]{./images/BLE_Stack.png}
		\caption{Pila de BLE \cite{ble_stack}}
		\label{ble_stack}
	\end{center}
\end{figure}

La pila pròpia de Bluetooth Low Energy esta dividida en dues parts, el controladors i el \textit{host}. Aquestes dues parts són independents i utilitzen el protocol HCI (\textit{Host Controller Interface}) per comunicar-se entre si. Aquest protocol pot estar implementat amb qualsevol protocol de transport físic com USB o UART. També hi ha la possibilitat d'implementar les dues parts en un mateix xip. L'arquitectura de si està implementat amb un o dos components s'anomena configuració de xip únic o dual.


\subsection{Controller}
El controlador compren la capa física i la capa d'enllaç.

\subsubsection{PHY}
La capa física és la que s'encarrega de la comunicació anal·lògica modulant i desmodulant les senyals.
Tal i com ja s'ha comentat abans treballa a la banda de 2.4 GHz en 40 canals diferents

\begin{figure}[h!]
	\begin{center}
		\includegraphics[width=0.8\textwidth]{./images/ble_channel_assignment.png}
		\caption{Pila de BLE \cite{ble_stack}}
		\label{ble_stack}
	\end{center}
\end{figure}

Els canals es classifiquen com a dades o d'anunci (\textit{Advertisment}). Els canals d'advertisment que són els que es vol tenir més qualitat coincideixen en els espais freqüencials que minimitzen les interferències amb els canals wifi més comuns (1, 6 i 11).

Tot i que amb aquest protocol l'objectiu és transmetre poca informació de forma eficient, es per això que, tot i estar tractant amb paquets d'informació molt curts cal una tassa de transmissió el més alta possible. D'aquesta manera s'aconsegueix que les radio tant del transmissor com la dels receptors estiguin el mínim temps possible actives.

Pel que fa a la potencia de transmissió es pot controlar de tal manera que si és necessari i no limita la comunicació es pot disminuir per gastar menys energia. Tot i així, en el paquets que s'envien es menciona la potència amb la que s'ha transmès per a que el receptor pugui estimar la distància fins al transmissor(?).

La modulació utilitzada per BLE és la mateixa que pel IEEE 802.11, GFSK (\textit{Gaussian Frequency Shift Keying}). Aquesta modulació és una de les més robustes, simples d'implementar i és el que permet (en part) a BLE tenir un abast molt més gran que Bluetooth Clàssic. Aplicar el filtre gaussià és útil ja que permet reduir el consum pic de energia \cite{BLE_Review} i també redueix les interferències en freqüències veïnes.
En Bluetooth 4.0 s'utilitza una desviació freqüencial de 185 kHz, en BLE 5 com que augmenta la velocitat de símbols també creix la interferència intersimbòlica. Per mitigar aquest efecte negatiu la desviació de freqüència passa a ser de 370 kHz


\subsubsection{Link Layer}

La capa d'enllaç és la encarregada de escanejar, anunciar i gestionar connexions amb altres dispositius

Per mitigar l'efecte de les interferències el protocol utilitza salts en freqüència (\textit{frequency hopping}). Això permet reduir l'impacte d'una interferència estreta (?). Els salts són de entre 5 i 16 canals d'entre els 37 dedicats a dades.

Bluetooth també permet la implementació de salts en freqüència adaptatius on només s'utilitzen canals suficientment bons descartant aquells en que es considera que hi ha masses interferències.

\subsection{Host}
\subsubsection{L2CAP}
La \textit{Logical Link Control and Adaptation Protocol} és la capa encarregada de l'establiment de la connexió lògica, multiplexament de protocols, segmentació i 'reasembly', control de flux per cada canal L2CAP.

La multiplexació de protocols és necessària tal que serveix per identificar quin és el protocol que s'utilitza en les capes superiors.

Per limitació física de l'arquitectura existeix una MTU (\textit{Maximum Transmission Unit}) i per tant els paquets de les capes superiors es converteixen en paquets més petits per a les capes inferiors.
Aquesta MTU es pot definir per cada connexió així flexibilitzant la varietat de dispositius amb els que es pot connectar un mateix dispositiu.

QOS Aquesta capa també és la que fa el seguiment de la qualitat de la connexió i dels recursos utilitzats per assegurar-se que les necessitats dels serveis es compleixen.

\subsubsection{SMP}
La capa \textit{Security Management Protocol} proveeix de diferents serveis relacionats amb la seguretat de la connexió.
Aquests serveis són: autenticació i autorització de dispositius i també integritat, confidencialitat i privacitat de les dades.
El protocol té tipus d'emparellament i generació de claus flexible per aconseguir reduir els requeriments de memòria i energia.

\subsubsection{ATT \& GATT}
El \textit{Attribute Protocol} és el protocol d'aplicació més comú per a BLE i el \textit{Generic Attribute Profile} defineix com utilitzar el protocol per oferir serveis a capes superiors.
El ATT és un protocol dissenyat per a dispositius \textit{Low Energy} amb l'objectiu de minimitzar la quantitat de dades transmeses. El atribut està format per 4 elements, \textit{handle}, UUID, permisos  i \textit{value}.

El \textit{handle} fa la distinció única entre els diferents atributs, ocupa 16 bits i no és obligatori que siguin valors seqüencials. És molt útil ja que s'utilitza per referenciar el atribut amb el mínim de bits possibles.
El UUID (\textit{Universal Unique IDentifier}) identifica el tipus d'atribut, aquest numero pot ser de 16 bits si s'utilitza algun que ja estandarditzat pel SIG o bé en tindrà 128 bits si està definit pel fabricant.
En els permisos s'indicarà quin tipus d'accés té el client a la informació (només lectura, lectura i escriptura ...). També pot estar definit si requereix un nivell mínim d'encriptació o si es necessari l'autenticació.
Per últim el valor del atribut serà on hi ha la informació en si i la seva longitud i interpretació dependrà del UUID tot i que té un limit de 512 bytes.

Des del punt de vista del protocol els dispositius són clients i servidors, normalment el client pren la iniciativa demanant dades però el servidor també te la capacitat d'iniciar una comunicació per exemple notificant quan un valor ha canviat.

La definició del ATT és massa genèrica per si sola tal que seria comú que per fer el mateix es desenvolupessin múltiples definicions que fossin incompatibles entre sí.
Per tal de tenir millor definits els serveis s'utilitza el GATT. El GATT permet definir perfils que agrupen múltiples atributs en un sol servei \cite{services}.

\begin{figure}[h!]
	\begin{center}
		\includegraphics[width=0.3\textwidth]{./images/GATT_Hierarchy.png}
		\caption{Jerarquia de GATT \cite{GATT_Hierarchy}}
	\end{center}
\end{figure}

En un llistat de atributs el GATT identifica els serveis tenint en compte que cada servei comença amb un atribut amb el UUID 0x2800, els següents atributs formaran part del mateix servei fins que es trobi un nou atribut amb UUID 0x2800. En el valor del atribut on es defineix el servei (UUID 0x2800) hi ha un altre UUID que especifica quin tipus de servei és.

Cada servei té característiques \cite{characteristics} que s'identifiquen perquè el UUID és 0x2803, en aquests atributs en el seu valor hi haurà un nou UUID que identifica la informació que es troba en el atribut amb un cert \textit{handle} que també estarà indicat.

\begin{center}
	\begin{tabular}{|l|l|l|l|}
		\hline
		Handle	&	UUID	&	Descripció						&	Valor		\\ 	\hline
		0x0100	&	0x2800	&	Battery Service					&	UUID 0x180F	\\		\hline
		0x0101	&	0x2803	&	Characteristic: Battery Level	&	\parbox[t]{4cm}{UUID 0x2A19	\\ Value handle: 0x0102}	\\	\hline
		0x0102	&	0x2A2B	&	Battery Value					&	20	\\	\hline
		0x0103	&	0x2800	&	Custom Temperature Service		&	UUID 	706676c8-3e49...	\\	\hline
		0x0104	&	0x2803	&	Characteristic: Temperature		&	\parbox[t]{4cm}{UUID 0x2A6E	\\ Value handle: 0x0105}	\\		\hline
		0x0105	&	0x2A6E	&	Temperature Value				&	25.45	\\	\hline
		0x0106	&	0x2803	&	Characteristic: date/time		&	\parbox[t]{4cm}{UUID 0x2A08	\\ Value handle: 0x0107}	\\		\hline
		0x0107	&	0x2A08	&	Date/Time						&	1/1/1980 12:00	\\
		\hline
	\end{tabular}

Exemple de possibles atributs
\end{center}

En aquest exemple es pot veure que tenim 2 serveis diferents ja que hi ha 2 UUIDs 0x2800 i tenim 3 característiques en total (una pel primer servei i dues pel segon) ja que hi ha 3 atributs amb UUID 0x2803.
Com que els calors corresponents al nivell de bateria i la temperatura estan estandarditzats no cal especificar a que es refereixen a percentatge de bateria restant i a graus Celsius.

Es pot veure com el servei que està definit per a la temperatura no forma part de l'estàndard ja que té 128 bits. El número que s'ha escollit és un UUID completament aleatori: 706676c8-3e49-4ecc-9379-fa9851444e53. Tot i que no hi ha una coordinació per assegurar-se que diferents desenvolupadors no utilitzen el mateix UUID degut a la longitud (128 bits) es considera improbable. La condició que ha de complir el UUID és que no sigui XXXXXXXX-0000-1000-8000-00805F9B34FB ja que aquest sufix correspon als que estan reservats per a l'estandard. En cas de voler tenir un UUID global reservat es pot fer pagant \$2.500 i també es poden veure tots els que ja s'han reservat per a empreses a questa web \cite{reservedUUIDs}.

En aquest exemple no n'hi ha cap però les característiques poden tenir descriptors \cite{descriptors} que permeten aportar informació addicional sobre la característica que els precedeix.

\subsubsection{GAP}
En quant es realitza una connexió els dispositius han de definir-se entre els següents rols: Anunciador (\textit{Advertiser}) o Escàner (\textit{Scanner}), Esclau (\textit{Slave}) o Mestre (\textit{Master}) i Emissor (\textit{Broadcaster}) o Observador (\textit{Observer}).
Aquests rols són independents per cada connexió per tant un dispositiu pot ser mestre en una connexió i alhora esclau en una altre.

Per iniciar una connexió entre dos dispositius (\textit{Peer-to-Peer}) un dispositiu vol ser descobert i envia missatges anunciant-se. L'altre dispositiu, que es vol connectar passa a ser un escàner. Aquest envia un paquet amb el requeriment de connexió, un cop acceptat, el dispositiu que fa d'escàner passa a ser mestre i el que anunciava passa a ser esclau.

També és possible transmetre informació des d'un dispositiu a tots aquells que estiguin escoltant, per tant, una comunicació de un a molts (\textit{One-to-many}). En aquest cas aquell que vol transmetre informació és el emissor i els dispositius que escolten són observadors.

\begin{figure}[h]
	\begin{center}
		\includegraphics[width=1\textwidth]{./images/rols_unicast.png}
		\caption{Pila de BLE \cite{ble_stack}}
		\label{ble_stack}
	\end{center}
\end{figure}


\chapter{Pràctica}
\section{Placa base CC1352R1}
Per analitzar el protocol BLE i veure les seves característiques s'ha utilitzat el kit per desenvolupament ràpid del microcontrolador CC1352R1.
\begin{figure}[h!]
	\begin{center}
		\includegraphics[width=0.8\textwidth]{./images/launchxl-cc1352r1.jpg}
		\caption{Placa \cite{placa}}
	\end{center}
\end{figure}

Aquesta placa permet el desenvolupament d'aplicacions en BLE utilitzant el microcontrolador CC1352 de Texas Instruments.
La placa té les següents característiques:


\section{Software}
Per al desenvolupament de projectes per a la placa s'ha utilitzat el entorn de desenvolupament Code Composer Studio. El software de desenvolupament és el SimpleLink(TM) CC13X2
Comentar tema versió

\section{Project 0}
El Project 0 és el projecte instal·lat amb que les plaques vénen de fàbrica. Aquest projecte exposa certs serveis a través de BLE i et permet veure una comunicació simple entre la placa i un dispositiu mòbil. La comunicació es fa a través del control dels dos LEDs que te la placa (els LEDs que estan directament controlars pel microcontrolador) i també per l'estat dels botons que hi ha a ambdós costats de la placa.
*Aquest projecte també inclou altres serveis en que no entrarem.



\section{Coses per defecte}
\subsection{Fórmules, figures i taules}

\subsubsection{Fórmules}

Pel que fa al format de les formules, no cal fer absolutament res. \LaTeX \ ja ho fa tot per vosaltres :-)  

Simplement el que heu de fer és escriure la fórmula com per exemple:

\begin{equation}\label{E:prova}
\sum _{i=0}^{N} \sum _{j=0}^{N} \frac{\lambda _{ij}}{\alpha _i \beta_ j} \cdot \cos (2\pi f_i) \sin(2 \pi f_j)
\end{equation}

i per citar-la és tant fàcil com fer: \ref{E:prova}

\subsubsection{Figures}

Pel que fa al format de les figures, no cal fer absolutament res. \LaTeX \ ja ho fa tot per vosaltres :-) . 

Simplement el que heu de fer és inserir la figura amb el codi següent:

\begin{verbatim}
\begin{figure}[htb]
\begin{center}
\includegraphics[width=0.5\textwidth]{./setup/EETAC-positiu-negre}
\caption{Exemple de figura}
\label{F:prova}
\end{center}
\end{figure}
\end{verbatim}

donant com a resultat la figura \ref{F:prova}.

\begin{figure}[htb]
\begin{center}
\includegraphics[width=0.5\textwidth]{./setup/EETAC-positiu-negre}
\caption{Exemple de figura}
\label{F:prova}
\end{center}
\end{figure}

Podem tocar la variable \texttt{width} per ajustar l'amplada de la figura com més ens convingui. Teniu en compte que la variable \texttt{textwidth} guarda el valor de l'amplada del texte dins la pàgina i per tant és una bona referència per delimitar amplades de figura, així doncs, la figura \ref{F:prova} ocupa la meitat de l'amplada del texte en una pàgina. 

Podeu arranjar múltiples imatges en una sola figura amb el paquet \texttt{subfigmat}, que defineix un nou entorn \texttt{subfigmatrix} que accepta per argument el nombre de columnes del arranjament. Per exemple, per fer una figura amb tres columnes d'imatges:

\begin{verbatim}
\begin{figure}[htb]
  \begin{center}
    \begin{subfigmatrix}{3}
      \subfigure[Títol subfigura 1]
         {\includegraphics{./setup/EETAC-positiu-negre}\label{SF:S1}} 
      \subfigure[Títol subfigura 2]
         {\includegraphics{./setup/EETAC-positiu-negre}\label{SF:S2}} 
      \subfigure[Títol subfigura 3]
         {\includegraphics{./setup/EETAC-positiu-negre}\label{SF:S3}} 
      \subfigure[Títol subfigura 4]
         {\includegraphics{./setup/EETAC-positiu-negre}\label{SF:S4}} 
      \subfigure[Títol subfigura 5]
         {\includegraphics{./setup/EETAC-positiu-negre}\label{SF:S5}} 
      \subfigure[Títol subfigura 6]
         {\includegraphics{./setup/EETAC-positiu-negre}\label{SF:S6}} 
    \end{subfigmatrix}
    \caption{Exemple d'arranjament amb múltiples imatges}
    \label{F:prova2}
  \end{center}
\end{figure}
\end{verbatim}

que dona com a resultat, la figura \ref{F:prova2}.

\begin{figure}[htb]
  \begin{center}
    \begin{subfigmatrix}{3}
      \subfigure[Títol subfigura 1]{\includegraphics{./setup/EETAC-positiu-negre}\label{SF:S1}} 
      \subfigure[Títol subfigura 2]{\includegraphics{./setup/EETAC-positiu-negre}\label{SF:S2}} 
      \subfigure[Títol subfigura 3]{\includegraphics{./setup/EETAC-positiu-negre}\label{SF:S3}} 
      \subfigure[Títol subfigura 4]{\includegraphics{./setup/EETAC-positiu-negre}\label{SF:S4}} 
      \subfigure[Títol subfigura 5]{\includegraphics{./setup/EETAC-positiu-negre}\label{SF:S5}} 
      \subfigure[Títol subfigura 6]{\includegraphics{./setup/EETAC-positiu-negre}\label{SF:S6}} 
    \end{subfigmatrix}
    \caption{Exemple d'arranjament amb múltiples imatges}
    \label{F:prova2}
  \end{center}
\end{figure}

%%%%%%%%%%%%%%%%%%%%%%%%%%%%%%%%%%%%%%%%%%%%%%%%%%%%%%%%%%%%%%%%%%%%%%%%%%%%%%


\subsubsection{Taules}

Pel que fa a la numeració de les taules, no cal fer gran cosa. \LaTeX \ ja fa gran part de la feina bruta. 

Simplement el que heu de fer és inserir la figura amb el codi següent:

\begin{verbatim}
\begin{table}[htb]
\begin{center}
\begin{tabular}{|c|l|r|}
\hline
{\bf Títol de la Columna 1} & {\bf Títol de la Columna 2} & 
{\bf Títol de la Columna 3}  \\ \hline \hline
centrada        & a l'esquerra    & a la dreta       \\ \hline
centrada        & a l'esquerra    & a la dreta       \\ \hline
centrada        & a l'esquerra    & a la dreta       \\ \hline
centrada        & a l'esquerra    & a la dreta       \\ \hline
centrada        & a l'esquerra    & a la dreta       \\ \hline
\end{tabular}
\caption{Exemple de taula}
\label{T:prova}
\end{center}
\end{table}
\end{verbatim}

donant com a resultat la taula \ref{T:prova}.


\begin{table}[htb]
\caption{Exemple de taula}
\begin{center}
\begin{tabular}{|c|l|r|}
\hline
{\bf Títol de la Columna 1} & {\bf Títol de la Columna 2} & {\bf Títol de la Columna 3}  \\ \hline \hline
centrada        & a l'esquerra    & a la dreta       \\ \hline
centrada        & a l'esquerra    & a la dreta       \\ \hline
centrada        & a l'esquerra    & a la dreta       \\ \hline
centrada        & a l'esquerra    & a la dreta       \\ \hline
centrada        & a l'esquerra    & a la dreta       \\ \hline
\end{tabular}
\label{T:prova}
\end{center}
\end{table}

L'entorn \texttt{tabular} que ofereix \LaTeX \ és molt complet i permet crear multitud de taules diferents, tot i que és alhora bastant complexe. Cau fora de les intencions del present document descriure la sintaxis i el format d'aquest tipus d'entorn. És molt fàcil trobar informació al respecte amb llibres especialitzats o simplement a Internet. 


\subsection{Estudi d'ambientalització}

En general l'estudi d'ambientalització es podrà incloure dins de la introducció o a les conclusions, llevat del cas que les repercussions ambientals del treball tinguin una importància tant rellevant que sigui recomanable dedicar-hi un capítol específic.


\subsection{Bibliografia}

Tret que el treball consisteixi en la cerca de bibliografia sobre un tema concret, la bibliografia ha de contenir només la llista d'obres consultades.

A la bibliografia s'han de llistar conjuntament llibres i articles de revistes. Citar una referència bibliogràfica és tant fàcil com fer:

\begin{verbatim}
\cite{prova1}
\end{verbatim}

per citar la referència \cite{prova1}.

El format de la bibliografia es genera automàticament. Un altre (gran!) avantatge del \LaTeX \ :-)


\section{Apèndixs}

En general, cal posar als apèndixs totes les dades i documents que farien el text feixuc i dificultarien la seva lectura, sense oblidar que les contínues referències als apèndixs poden obligar al lector a interrompre constantment la lectura del treball. És per això, que els apèndixs poden incloure diagrames, dades estadístiques, taules de resultats i desenvolupaments teòrics complementaris.

En el cas que, amb el conjunt d'apèndixs, el treball tingui una extensió superior als 100 fulls (aproximadament 200 pàgines), els apèndixs s'han de enquadernar en un volum separat del cos principal del treball. Aquesta plantilla ja proporciona les eines necessàries per fer la portada necessària en cas d'enquadernar els apèndixs per separat.

\chapter{Experimental}
\section{ADC}
\section{Range}
\section{Throughput}
\chapter{Projecte de sensors}
\section{Mesures Biològiques}
\section{Mesures mediambientals}
\subsection{Arquitectura}
\subsection{Estudi energètic}
\cleardoublepage
\phantomsection
\chapter*{Conclusions}

Aquest projecte serveix per considerar si BLE és la tecnologia adequada per a les aplicacions de mesures mediambientals o biològiques.

\section*{Taxa de dades necessària}

Un dels dubtes a l'inici del treball era si aquesta tecnologia tenia una capacitat de transmissió suficient per a xarxes de sensors.
Tot i que el Bluetooth Low Energy té bastant menys capacitat per transmetre dades comparant amb tecnologies com wifi, les dades necessàries en aquest tipus de xarxes acostumen a ser molt reduïdes.
La tecnologia BLE no és capaç de transmetre vídeo o àudio en temps real però és suficient per transmetre fluxos de dades de sensors amb mesures biològiques o mediambientals.

\section*{Abast}

Com s'ha comentat l'abast de la tecnologia BLE ve limitat tant per la distància entre transmissor i receptor com per les interferències.
En mesures biològiques es pot considerar que el transmissor i el receptor estaran relativament a prop per tant no hi haurà problemes d'abast.
En canvi per a mesures ambientals si considerem instal·lacions privades en habitatges, cal tenir en compte els obstacles que s'oposen al senyal.
També cal considerar la interferència més que probable de les tecnologies amb les quals BLE comparteix espectre amb wifi, altres dispositius intel·ligents i amb Bluetooth Clàssic mateix.
Per combatre aquests desavantatges serà suficient augmentant la potència de transmissió tot i que reduirà la vida de la bateria en cas que en tingui.

\section*{Energia Consumida}

BLE és dels protocols que permet consumir menys energia per a transmetre informació.
Això només serà un avantatge quan els sensors estiguin alimentats per bateria i la transmissió sigui el que consumeix més energia de tot el sistema.
Si els sensors tenen pantalles, s'alimenten de la instal·lació elèctrica, fan un processament de les dades o emmagatzemen les dades localment, és possible que el consum de la transmissió no sigui significant.
En aquests casos no s'ha d'escollir el protocol de transmissió segons el consum, ja que el temps entre recàrregues no es veu afectat pel protocol.


\section*{Penetració del mercat}

És evident que la presència de BLE en tots els mòbils intel·ligents facilita l'ús i simplifica l'arquitectura d'un desplegament.
En cas de no utilitzar BLE no hi ha cap tecnologia de baix consum que es pugui connectar directament amb els telèfons dels usuaris.
Per tant, és necessari un dispositiu entremig entre els sensors i els usuaris que s'anomena concentrador o \textit{hub}.
En arquitectures simples suposa un cost extra però si es fa un desplegament extens, el cost és marginal.

En el cas de connexió directa fins al terminal de l'usuari BLE és perfecte, ja que es troba en tots els telèfons intel·ligents.
En total es preveu la producció de 7.500 milions de dispositius entre 2020 i 2024 amb un creixement del 26\% anual segons el SIG\cite{Bluetooth_Market_Update_2020}.


\cleardoublepage
\phantomsection
\begin{thebibliography}{2}

%% Llibres:  Autor/s (cognoms i inicials dels noms), títol del llibre (en cursiva), editor, ciutat i any de publicació. Quan es cita el capítol d'un llibre s'ha d'indicar el títol del capítol (entre cometes), el títol del llibre (en cursiva) i els números de pàgines amb la primera i la darrera incloses.

\bibitem{Original_BLE_Extension}
Honkanen, M.; Lappetelainen, A.; Kivekas, K. (2004). \textit{Low end extension for Bluetooth}. 2004 IEEE Radio and Wireless Conference 19–22 September 2004. IEEE. pp. 199–202.
[Consulta: Setembre 2020] \href{https://ieeexplore.ieee.org/document/1389107}{https://ieeexplore.ieee.org/document/1389107}

\bibitem{MIMOSA}
Arxiu de la web de MIMOSA.\newline
\href{https://web.archive.org/web/20160804035320/http://www.mimosa-fp6.com/}{https://web.archive.org/web/20160804035320/http://www.mimosa-fp6.com/}


\bibitem{BLE_5_improvement_over_4}
Millores de BLE 5 respecte BLE 4.\newline
\href{https://www.bluetooth.com/wp-content/uploads/2019/03/Bluetooth\_5-FINAL.pdf}{https://www.bluetooth.com/wp-content/uploads/2019/03/Bluetooth\_5-FINAL.pdf}

\bibitem{LoRaWan_Energy}
Casals Ibáñez, Lluis \& Mir Masnou, Bernat \& Vidal Ferré, Rafael \& Gomez, Carles. (2017). \textit{Modeling the energy performance of LoRaWAN}. Sensors. 17. 2364. 10.3390/s17102364.
[Consulta: Setembre 2020]
\href{https://www.researchgate.net/publication/320435869}{https://www.researchgate.net/publication/320435869}

\bibitem{802.15.4_throughput}
Latré, Benoît; De Mil, Pieter; Moerman, Ingrid; Dierdonck, Niek; Dhoedt, Bart; Demeester, Piet. (2005). \href{https://www.researchgate.net/publication/220963645_Maximum_Throughput_and_Minimum_Delay_in_IEEE_802154}{Maximum Throughput and Minimum Delay in IEEE 802.15.4.} 3794. 866-876. 10.1007/11599463\_84. 

\bibitem{Baud_definition}
Definició de Baud en el Diccionari de les Telecomunicacions.\newline
\href{https://www.termcat.cat/en/diccionaris-en-linia/235/fitxa/Njc0Njkz}{https://www.termcat.cat/en/diccionaris-en-linia/235/fitxa/Njc0Njkz}

\bibitem{BLE_Review}
Gomez, C.; Oller, J.; Paradells, J.  \href{https://www.mdpi.com/1424-8220/12/9/11734}{Overview and Evaluation of Bluetooth Low Energy: An Emerging Low-Power Wireless Technology}. Sensors 2012, 12, 11734-11753.

\bibitem{Link_Layer_states}
Link Layer States
\href{https://dev.ti.com/tirex/explore/content/simplelink_academy_cc13x2_26x2sdk_4_10_01_00/modules/ble5stack/ble_connections/ble_connections.html\#the-link-layer}{https://dev.ti.com/tirex/explore/content/simplelink\_academy\_cc13x2\_26x2sdk\_4\_10\_01\_00/modules/ble5stack/ble\_connections/ble\_connections.html\#the-link-layer}

\bibitem{GATT_Hierarchy}
O’Reilly online learning
Chapter 4. GATT (Services and Characteristics)
\href{https://www.oreilly.com/library/view/getting-started-with/9781491900550/ch04.html}{https://www.oreilly.com/library/view/getting-started-with/9781491900550/ch04.html}

\bibitem{services}
Bluetooth Special Interest Group
``GATT Services Specification``\newline
\href{https://www.bluetooth.com/specifications/gatt/services/}{https://www.bluetooth.com/specifications/gatt/services/}

\bibitem{characteristics}
Bluetooth Special Interest Group
``GATT Charasteristics Specification``\newline
\href{https://www.bluetooth.com/specifications/gatt/characteristics/}{https://www.bluetooth.com/specifications/gatt/characteristics/}

\bibitem{descriptors}
Bluetooth Special Interest Group
``GATT Descriptors Specification``\newline
\href{https://www.bluetooth.com/specifications/gatt/descriptors/}{https://www.bluetooth.com/specifications/gatt/descriptors/}

\bibitem{Battery_Level}
Especificació de la característica de nivell de bateria.\newline
\href{https://www.bluetooth.com/wp-content/uploads/Sitecore-Media-Library/Gatt/Xml/Characteristics/org.bluetooth.characteristic.battery\_level.xml}{https://www.bluetooth.com/wp-content/uploads/Sitecore-Media-Library/Gatt/Xml/Characteristics/org.bluetooth.characteristic.battery\_level.xml}

\bibitem{Temperature_Characteristic}
Especificació de la característica de temperatura. \newline
\href{https://www.bluetooth.com/wp-content/uploads/Sitecore-Media-Library/Gatt/Xml/Characteristics/org.bluetooth.characteristic.temperature.xml}{https://www.bluetooth.com/wp-content/uploads/Sitecore-Media-Library/Gatt/Xml/Characteristics/org.bluetooth.characteristic.temperature.xml}

\bibitem{reservedUUIDs}
Llistat de UUIDs reservats per a empreses.\newline
\href{https://www.bluetooth.com/specifications/assigned-numbers/16-bit-UUIDs-for-Members/}{https://www.bluetooth.com/specifications/assigned-numbers/16-bit-UUIDs-for-Members/}

\bibitem{Advertising}
https://www.bluetooth.com/blog/bluetooth-low-energy-it-starts-with-advertising/

\bibitem{advertisment_params}
\href{https://dev.ti.com/tirex/explore/content/simplelink\_academy\_cc13x2\_26x2sdk\_4\_10\_01\_00/modules/ble5stack/ble\_scan\_adv\_basic/ble\_scan\_adv\_basic.html\#advertising-parameter}{https://dev.ti.com/tirex/explore/content/simplelink\_academy\_cc13x2\_26x2sdk\_4\_10\_01\_00/modules/ble5stack/ble\_scan\_adv\_basic/ble\_scan\_adv\_basic.html\#advertising-parameters}

\bibitem{fig:connection_establishement}
\href{https://microchipdeveloper.com/wireless:ble-link-layer-connections}{https://microchipdeveloper.com/wireless:ble-link-layer-connections}

\bibitem{slave_latency}
\href{http://software-dl.ti.com/lprf/simplelink\_cc2640r2\_latest/docs/blestack/ble\_user\_guide/html/ble-stack-3.x/gap.html\#connection-parameters}{http://software-dl.ti.com/lprf/simplelink\_cc2640r2\_latest/docs/blestack/ble\_user\_guide/html/ble-stack-3.x/gap.html\#connection-parameters}

\bibitem{BLE_4.2_packet_format}
BLUETOOTH SPECIFICATION Version 4.2 [Vol 6, Part B] apartat 2

\bibitem{AD_Types}
\href{https://www.bluetooth.com/specifications/assigned-numbers/generic-access-profile/}{https://www.bluetooth.com/specifications/assigned-numbers/generic-access-profile/}

\bibitem{BLE_5_Extended_Advertising}
BLUETOOTH CORE SPECIFICATION Version 5.2 | Vol 6, Part B apartat 2.3.4

\bibitem{placa}
%TODO: Indicar la url amb la web de la foto
\href{http://www.ti.com/diagrams/med_launchxl-cc1352r1_launchxl-cc1352r1.jpg}{http://www.ti.com/diagrams/med\_launchxl-cc1352r1\_launchxl-cc1352r1.jpg}

\bibitem{placa_datasheet}
CC1352R SimpleLink Datasheet\newline
%TODO: Link not working
\href{http://www.ti.com/lit/ds/swrs196f/swrs196f.pdf}{http://www.ti.com/lit/ds/swrs196f/swrs196f.pdf}

\bibitem{extended properties}
Characteristic Extended Properties\newline
\href{https://www.bluetooth.com/wp-content/uploads/Sitecore-Media-Library/Gatt/Xml/Descriptors/org.bluetooth.descriptor.gatt.characteristic_extended_properties.xml}{https://www.bluetooth.com/wp-content/uploads/Sitecore-Media-Library/Gatt/Xml/Descriptors/org.bluetooth.descriptor.gatt.characteristic\_extended\_properties.xml}

\bibitem{serial_params}
\href{http://dev.ti.com/tirex/explore/content/simplelink_academy_cc13x2_26x2sdk_4_10_01_00/modules/ble5stack/ble_01\_basic/resources/btool_serial_port_config.png}{http://dev.ti.com/tirex/explore/content/simplelink\_academy\_cc13x2\_26x2sdk\_4\_10\_01\_00/modules/ble5stack/ble\_01\_basic/resources/btool\_serial\_port\_config.png}

\bibitem{Service_Generator}
http://dev.ti.com/tirex/explore/content/simplelink\_academy\_cc13x2\_26x2sdk\_4\_10\_01\_00/modules/ble5stack/ble\_01\_custom\_profile/ble\_01\_custom\_profile.html\#example-service-generator

\bibitem{Bluetooth_Market_Update_2020}
https://www.bluetooth.com/wp-content/uploads/2020/03/2020\_Market\_Update-EN.pdf

\bibitem{Bluetooth Specification}
Bluetooth Special Interest Group
``Bluetooth Core Specification``
(Revision 5.2)

\bibitem{Diccionari_Telecomunicacions}
UNIVERSITAT POLITÈCNICA DE CATALUNYA; TERMCAT, CENTRE DE TERMINOLOGIA; ENCICLOPÈDIA CATALANA. Diccionari de telecomunicacions [en línia]. Barcelona: TERMCAT, Centre de Terminologia, cop. 2017. (Diccionaris en Línia) (Ciència i Tecnologia)
<https://www.termcat.cat/ca/diccionaris-en-linia/235>

\end{thebibliography}

%%%%%%%%%%%%%%%%%%%%%%%%%%%%%%%%%%%%%%%%%%%%%%%%%%%%%%%%%%%%%%%%%%%%%%%%%%
%%%%%%                           APENDIXS                         %%%%%%%%
%%%%%%%%%%%%%%%%%%%%%%%%%%%%%%%%%%%%%%%%%%%%%%%%%%%%%%%%%%%%%%%%%%%%%%%%%%
\pagestyle{empty}  % no tocar
 
%% Descomentar una de les dues línies següents, en funció de:
%%  a) els apendixs s'encuadernaran apart (amb portada) 
%%  b) els apendixs s'enquadernen amb el mateix projecte (sense portada). 
%% Recordeu que si tot el document (amb apèndixs) excedeix les 100 pagines 
%% s'ha d'enquadernar a part
%\appendix\ambportada
\appendix\senseportada


%%%%%%%%%%%%%%%%%%%%%%%%%%%%%%%%%%%%%%%%%%%%%%%%%%%%%%%%%%%%%%%%%%%%%%%%%%
%%%%%% INCLOURE A PARTIR D'AQUI TOTS ELS CAPÍTOLS DELS APENDIXS   %%%%%%%%
%%%%%%%%%%%%%%%%%%%%%%%%%%%%%%%%%%%%%%%%%%%%%%%%%%%%%%%%%%%%%%%%%%%%%%%%%%

\chapter{Datasheet del LaunchXL CC1352R1}
\label{datasheet}

\begin{figure}[h!]
	\begin{center}
		\includegraphics[width=\textwidth]{./images/pinout.PNG}
			\caption{Manual del LaunchXL}
	\end{center}
\end{figure}

La figura correspon al manual que ve amb el kit de desenvolupament i descriu quines funcionalitats té cada pin de la PCB.


\chapter{Valors del multímetre i l'ADC}
\label{raw_adc}

Valors mesurats amb l'ADC i el multímetre en el mateix instant en mil·livolts.

\begin{table}[!h]
	\begin{center}
		\begin{tabular}{|c|c|}
			\hline
			Multímetre			&	PCB		\\	\hline
			2.2			&	5.2		\\	\hline
			420			&	417		\\	\hline
			738			&	727		\\	\hline
			1218			&	1190		\\	\hline
			1493			&	1517		\\	\hline
			1858			&	1817		\\	\hline
			2300			&	2236		\\	\hline
			2730			&	2691		\\	\hline
			3220			&	3128		\\	\hline
		\end{tabular}
	\end{center}
	\caption{Taula dels voltatges mesurats}
\end{table}


\end{document}






