\chapter{Projecte de sensors}
\section{Experimentació}
Per comprendre millor com afecten els diferents paràmetres configurables corresponents a BLE i també les prestacions dels perifèrics de la placa es realitzaran els següents escenaris.

\subsection{ADC}
La placa té un convertidor analògic-digital que ens permetrà enviar senyals analògiques un cop s'han mostreajat.
Aquest ADC té x canals i una freqüència de mostreig de 200 Kmostres per segon.
Com que té x canals es poden 

S'analitza a biologiques


\subsection{Range}

S'analitza a mediambientals

\subsection{Throughput}

s'analitza a biologiques

\subsection{Consum d'Energia}

S'analitza a mediambientals


\section{Mesures Biològiques}

Si es volen transmetre la velocitat dels batecs del cor això suposa, aproximadament, una mostra cada segon d'uns 10 bits més un instant de temps de 30 bits suposa uns 40 bits/segon, insignificant.

Si el que es vol es transmetre el senyal directament que prové del sensor que mesura la sang.
Considerem el màxim de 200 batecs per minut, això és una freqüència de 3,3 Hz, si mostregem a 10 vegades Fmax, 33 mostres/segon amb una resolució de 10 bits la tassa es de 330 bits/s

\section{Throughput}
\section{ADC}


\section{Mesures mediambientals}
\subsection{Arquitectura}
\subsection{Estudi energètic}
\section{Abast}