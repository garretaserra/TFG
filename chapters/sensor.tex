\chapter{Projecte de sensors}
\section{Experimentació}
Per comprendre millor com afecten els diferents paràmetres configurables corresponents a BLE i també les prestacions dels perifèrics de la placa es realitzaran els següents escenaris.

\subsection{ADC}
La placa té un convertidor analògic-digital que ens permetrà enviar senyals analògiques un cop s'han mostrejat.
Aquest ADC té x canals i una freqüència de mostreig de 200 Kmostres per segon.
Com que té x canals es poden 

En aquest treball no es tractaran directament els sensors sinó que es simularan le senyals que en sortirien.
Com que la placa i el seu ADC treballa a 3.3V es crearà un circuit que permeti controlar un voltatge d'entre 0 i 3.3V.

Aquest circuit es basarà en un divisor de voltatge simple format per una resistència i un potenciòmetre.
El potenciòmetre permet canviar la resistència del element a través d'un cargol i junt amb el circuit que l'envolta permetrà canviar el voltatge a la sortida.
\begin{figure}[!h]
	\begin{center}
		\begin{circuitikz}
			\draw
			(0,2) node[anchor=east] {$V_{in}$}
			to [R=$R_1$, *-] (2,2)
			to [vR=$R_{pot}$, *-] (2,0) node[tlground](GND){};
			\draw
			(2,2) to [short, -*] (3,2)
			to (3,2) node[anchor=west] (3,2) {$V_{out}$};
		\end{circuitikz}

	\end{center}
\end{figure}

Aquest circuit segueix la estructura d'un divisor de voltatge per tant es pot calcular el voltatge a la sortida segons la següent fórmula.

\begin{equation}
	V_{in}\cdot\frac{R_{pot}}{R_1+R_{pot}}=V_{out}
\end{equation}

Per escollir els components cal tenir in compte utilitzar resistències de valor alt per reduir el consum d'energia.
S'utilitzarà un potenciòmetre de 10KOhms per tant serà una resistència que es podrà modificar des de 0 Ohms fins 10 KOhms.
Per trobar la $R_1$ que compleixi els requisits queda la següent fórmula.

\begin{equation}
R_1=5V\cdot\frac{10k\Omega}{3.3V}-10k\Omega\approx5151\Omega
\end{equation}

Per a la $R_1$ s'utilitzaran resistències de la sèrie E12, els valors que més s'acosten a $5121\Omega$ són $5.6k\Omega$ i $4.6k\Omega$.
Per assegurar-se que no es superen els 3.3V a la entrada del ADC que podria fer malbé el dispositiu s'escull la resistència superior, per tant el circuit final quedarà de la següent manera.

\begin{figure}[!h]
	\begin{center}
		\begin{circuitikz}
			\draw
			(0,2) node[anchor=east] {$5\,V$}
			to [R=$5.6\;k\Omega$, *-] (2,2)
			to [vR=$ \lbrack 0-10 \rbrack \;k\Omega$, *-] (2,0) node[tlground](GND){};
			\draw
			(2,2) to [short, -*] (3,2)
			to (3,2) node[anchor=west] (3,2) {$[0-3.3]\;V$};
		\end{circuitikz}

	\end{center}
\end{figure}


\subsection{Range}

S'analitza a mediambientals

\subsection{Throughput}

s'analitza a biologiques

\subsection{Consum d'Energia}

S'analitza a mediambientals


\section{Mesures Biològiques}

Si es volen transmetre la velocitat dels batecs del cor això suposa, aproximadament, una mostra cada segon d'uns 10 bits més un instant de temps de 30 bits suposa uns 40 bits/segon, insignificant.

Si el que es vol es transmetre el senyal directament que prové del sensor que mesura la sang.
Considerem el màxim de 200 batecs per minut, això és una freqüència de 3,3 Hz, si mostregem a 10 vegades Fmax, 33 mostres/segon amb una resolució de 10 bits la tassa es de 330 bits/s

\section{Throughput}
\section{ADC}


\section{Mesures mediambientals}
\subsection{Arquitectura}
\subsection{Estudi energètic}
\section{Abast}