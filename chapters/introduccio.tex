\cleardoublepage
\phantomsection
\chapter*{Introducció}
L'arquitectura Internet de les Coses (Internet of Things, IoT) que s'ha popularitzat en els últims anys ha comportat unes noves necessitats en molts aspectes de la tecnologia que són molt diferents dels anteriors.
Quan el que es vol és que tot estigui connectat entre si, queda clar que es trenquen moltes premisses amb les quals s'havien dissenyat originalment les tecnologies més importants que s'utilitzen avui en dia.

La transició cap a l'Internet of Things, transforma el paradigma de les comunicacions de pocs elements a molta velocitat (xarxes centralitzades) cap a molts elements amb poca velocitat (xarxes distribuïdes).
Això afecta les tecnologies sense fils que no havien estat dissenyades pel seu ús tan generalitzat.

Aquests canvis es poden veure en l'evolució de la majoria de tecnologies sense fils com wifi, xarxes cel·lulars i Bluetooth, entre d'altres.
En totes aquestes tecnologies ha estat necessari fer grans canvis per acomodar a molts més usuaris dels que s'havien predit inicialment.

El Bluetooth clàssic és un exemple d'una tecnologia que tenia limitacions a l'hora d'utilitzar-se en certs escenaris.
No permetia comunicació entre més d'un dispositiu alhora, tenia un abast limitat i poques mesures per contrarestar les interferències.
Al llarg de la seva història es van definir noves versions que anaven millorant les mancances del protocol.
Tot i això, el cost que suposa haver d'establir una connexió per transmetre dades, encara que siguin molt poques, segueix sent molt alt.

És per això que Bluetooth va incorporar l'extensió Low Energy de forma opcional.
En aquesta extensió està definit el protocol anomenat Bluetooth Low Energy que soluciona el problema de Bluetooth Clàssic, ja que, permet molt fàcilment i utilitzant molt poca energia transmetre dades (que inclou a múltiples dispositius) sense haver d'establir una connexió.

Però Bluetooth Low Energy no era perfecte i també va anar millorant amb les noves versions, augmentant el rendiment en escenaris concrets.
Aquestes millores inclouen augmentar el màxim de dades que es poden transmetre sense connexió o augmentar la taxa de dades de transmissió per poder estalviar energia.
En aquest treball es detallarà com funciona el BLE, com ha evolucionat al llarg del temps i finalment es realitzarà una implementació real del protocol.