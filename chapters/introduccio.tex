\cleardoublepage
\phantomsection
\chapter*{Introducció}
La arquitectura internet de les coses que s'ha popularitzat en els últims anys a comportat unes noves necessitats en molts aspectes de la tecnologia que són molt diferents als anteriors.
Quan el que es vol és que tot allò que es pugui, estigui connectat, queda clar que es trenquen moltes premisses amb les que s'havien dissenyat les tecnologies.

La transició cap a Internet Of Things, transforma el paradigma de les comunicacions de pocs elements a molta velocitat cap a molts elements amb poca velocitat
 Això afecta més a les tecnologies sense fils que no havien estat dissenyades amb aquestes noves necessitats.

Aquests canvis es poden veure en la evolució de la majoria de tecnologies sense fils com WiFi, xarxes cel·lulars i Bluetooth entre d'altres.
En aquest TFG s'avaluaran les noves versions de la tecnologia Bluetooth anomenades Bluetooth Low Energy.
Es desenvoluparà el disseny i s'implementarà la tecnologia BLE per a usos en casos reals de mesures mediambientals i biològiques.
L'objectiu és veure la viabilitat d'aquesta tecnologia en aquests casos oposats per entendre la flexibilitat que aporta el BLE.
Això es farà veient els avantatges i limitacions que té pel que fa al consum d'energia, resistència a interferències, abast i taxa de transmissió de dades.