\chapter{Origens del Bluetooth Low Energy}\label{C:compaginacio}

%\section{Ús lliure de RF}
%Amb l'establiment de les bandes ISM (Afegir referencia) es crea la possibilitat per el públic general utilitzar comunicació inal·làmbrica amb molta facilitat. Això es deu a que l'establiment de les bandes ISM està acceptat de manera (mes o menys(?)) de la mateixa manera internacionalment.
%Això facilita el disseny de productes per al consumidor habitual que ràpidament veu els avantatges de la comunicació inal·làmbrica entre dispositius.
%En aquest entorn sorgeix la necessitat d'establir estandards entre le companyies principals dels sectors. Per definir els estandards s'agrupen les companyies i formen grups com al WIFI Alliance o ...

%\subsection{Wifi}
%La tecnologia de comunicació més popular avui en dia és la WIFI, dissenyada originalment als anys (?) i establerta als anys (?) orientada a permetre una connexió (* bona) i sense consideració per les interferències entre si mateixa que no es podien preveure llavors.  
%- Avantatges
%- Innovació
%- Evolució

\section{Història de Bluetooth Clàssic}
El desenvolupament de tecnologia per a connexions de curt abast que acabarien esdevenint el que avui es coneix com a Bluetooth va començar al 1989 per part de Ericsson.
Inicialment la voluntat era utilitzar aquesta tecnologia per a auriculars inal·làmbrics. La tecnologia va ser primer utilitzada en els ordinadors portàtils de IBM (ThinkPad). Inicialment es volia integrar connectivitat de la xarxa de telefonia als ordinadors però no era factible degut al consum d'energia de la tecnologia mòbil.
IBM i Ericsson van acordar utilitzar aquesta tecnologia de curt abast per permetre a mòbils Ericsson i ordinadors ThinkPad comunicar-se entre si i poder accedir a la xarxa mòbil des d'un ordinador.
Com que ni Ericsson ni IBM tenien majoria en la quota de mercat dels respectius productes van decidir que la tecnologia fos oberta. Es van unir al grup Intel, Nokia i Toshiba, i totes 5 companyies el 1998 van fundar el \textit{Blueooth Special Interest group}.
El 2001 van sortir a la venta el primer mòbil amb Bluetooth, el Ericsson T39 i el primer portàtil el IBM ThinkPad A30

\section{Història de Bluetooth Low Energy}
El 2001 a Nokia es va començar a buscar una versió de Bluetooth que fos similar però que reduís significativament el consum d'energia amb el mínim de compromisos possibles.
El 2004 es va publicar la \textit{Bluetooth Low End Extension} \cite{Original_BLE_Extension}. 
Després de més desenvolupament juntament amb Logitech i MIMOSA [citation needed] es va divulgar al 2006 amb el nom de Wibree.

\begin{figure}[h]
	\begin{center}
		\includegraphics[width=0.6\textwidth]{./images/Wibree_Logo.png}
		\caption{Wibree Logo}
	\end{center}
\end{figure}

Després de negociar amb els membres del SIG es va acordar incloure Wibree al estandard de bluetooth en la especificació 4.0 amb el nom de \textit{Bluetooth ultra low power technology} i publicitat com a Bluetooth Smart. El primer mòbil a incloure'l va ser el iPhone 4S al 2011.

\section{Bluetooth vs BLE}
Bluetooth Low Energy no té cap relació amb Bluetooth Classic pel que fa a l'arquitectura del protocol. Tot i compartir nom (perque?) queda clar que no estem parlant del mateix amb el simple fet que no son compatibles entre sí. Tot i així comparteixen l'ús de la banda freqüencial de 2,4 GHz igual que altres protocols sense fils. Al 2016 es va anunciar la versió 5.0 anomenada Bluetooth 5 (sense el punt)[cita] i se li va canviar el nom (aclariment).


\section{MANETS}
Bluetooth clàssic no permetia tenir més d'una connexió establerta amb dispositius i l'arquitectura feia que s'utilitzés bastanta energia per establir i mantenir una connexió encara que es volguessin transmetre molt poques dades.
Això no era un problema inicialment ja que principalment l'ús era per la transferència de fitxers com contactes o fotografies.
Junt amb l'arribada dels telèfons intel·ligents els auriculars inal·làmbrics es van fer populars i també es va establir la dominància de Bluetooth per a escoltar música sense cables.
Amb aquest objectiu es va millorar Bluetooth per tal de ser més resistent a interferències i aconseguir velocitats superiors per poder transmetre música d'alta fidelitat.
Per un altre costat però va començar a sorgir el Internet of Things, basat en tenir molts dispositius connectats transmetent a taxes molt variades i de forma discontinua i això va generar necessitats que no es podien cobrir amb les tecnologies existents.
Era necessari tenir xarxes sense fils, amb consum molt baix d'energia i de llarg abast.

Amb aquests requeriments van sorgir les MANETS (Mobile Ad-Hoc Networks) i a continuació compararem les característiques de les tecnologies més importants amb BLE per veure on queda situat BLE.

\subsection{Versions de BLE}
En quant al estàndard Bluetooth Low Energy cal diferenciar les característiques de la versió 4.0 amb la 5 ja que hi ha canvis significatius \cite{BLE_5_improvement_over_4}.

\subsubsection{LE 2M}
En la versió de Bluetooth 5 permet a la capa física operar a 2 Msimbols/s enlloc dels 1 Ms/s que es podia anteriorment. L'augment de la velocitat dels símbols permet augmentar la quantitat de dades que es poden transmetre a nivell d'aplicació.
Aquest canvi està pensat per facilitar la utilització de BLE en situacions més exigents pel que fa a dades com per exemple múltiples mesures del cos humà.

\subsubsection{Abast}
Tot i que la tecnologia BLE ja de per si en la seva versió original (4.0) tenia una distància màxima de transmissió teòrica molt gran.
Però tenir la possibilitat de més abast en les comunicacions ajuden a cobrir les necessitats generades per sectors com el de les cases intel·ligents.
El límit del abast ve definit per la potència màxima de transmissió (sigui per protocol o per normativa) i la \textit{Bit Error Ratio} que s'accepta sent el percentatge de bits que es reben incorrectament.
Bluetooth defineix la BER mínima en 0.1\% per tant quant un de cada mil bits es erroni es considera que la connexió és massa dolenta i que ja s'ha arribat al límit de distància.
Per tractar els errors en transmissió hi ha dues estratègies: detecció d'errors i correcció d'errors.
En la versió 4.0 de Bluetooth només s'utilitzava la detecció d'errors.
En Bluetooth 5 s'implementa correcció d'errors d'aquesta manera sense augmentar la potència transmesa es pot incrementar el abast màxim que pot tenir la comunicació.
El desavantatge és que més bits dels que es transmeten es dediquen a corregir errors per tant queda menys espai per a dades així reduint el \textit{throughput}.


El esquema de correcció d'errors (\textit{Forward Error Correction}) es fa amb un codificador convolucional que pot generar a la sortida més bits dels que entren. El codificador pot funcionar amb dos esquemes diferents s'anomenen S=2 i S=8.
Amb els diferents esquemes s'utilitza un \textit{code rate} diferent que pot augmentar la capacitat de correcció i augmentar el abast però disminuir la quantitat de dades que es poden transmetre.

\begin{figure}[h!]
	\begin{center}
		\includegraphics[width=0.8\textwidth]{./images/LE_PHY.png}
		\caption{Comparació de diferents capes físiques}
	\end{center}
\end{figure}

\subsubsection{Advertisement Extensions}
\subsubsection{Slot Availability Masks}
\subsubsection{Improved Frequency Hopping}


\subsection{Altres Protocols}
Bluetooth Low Energy no és l'únic protocol orientat a la connexió de dispositius amb baix consum energètic. També existeixen els següents protocols:
Zigbee, utilitzat en Philips Hue
Zwave Ring (home security)
6lowpan
insteon
lorawan

% Prodcutes que utilitzen cadascun

\subsection{Comparació}
Tot i que els protocols tenen molt en comú cada un es diferencia de la resta en els detalls, a continuació veiem una comparació de les capacitats de cada un dels protocols

\section{BLE Stack}
Un cop vistes les característiques generals del protocol cal entendre com funciona per dins.
En entrar en detall queda clar com el protocol s'ha dissenyat de forma molt flexible 

\begin{figure}[h!]
	\begin{center}
		\includegraphics[width=0.6\textwidth]{./images/BLE_Stack.png}
		\caption{Pila de BLE \cite{ble_stack}}
		\label{ble_stack}
	\end{center}
\end{figure}

La pila pròpia de Bluetooth Low Energy esta dividida en dues parts, el controladors i el \textit{host}. Aquestes dues parts són independents i utilitzen el protocol HCI (\textit{Host Controller Interface}) per comunicar-se entre si. Aquest protocol pot estar implementat amb qualsevol protocol de transport físic com USB o UART. També hi ha la possibilitat d'implementar les dues parts en un mateix xip. L'arquitectura de si està implementat amb un o dos components s'anomena configuració de xip únic o dual.


\subsection{Controller}
El controlador compren la capa física i la capa d'enllaç.

\subsubsection{PHY}
La capa física és la que s'encarrega de la comunicació anal·lògica modulant i desmodulant les senyals.
Tal i com ja s'ha comentat abans treballa a la banda de 2.4 GHz en 40 canals diferents

\begin{figure}[h!]
	\begin{center}
		\includegraphics[width=0.8\textwidth]{./images/ble_channel_assignment.png}
		\caption{Pila de BLE \cite{ble_stack}}
		\label{ble_stack}
	\end{center}
\end{figure}

Els canals es classifiquen com a dades o d'anunci (\textit{Advertisment}). Els canals d'advertisment que són els que es vol tenir més qualitat coincideixen en els espais freqüencials que minimitzen les interferències amb els canals wifi més comuns (1, 6 i 11).

Tot i que amb aquest protocol l'objectiu és transmetre poca informació de forma eficient, es per això que, tot i estar tractant amb paquets d'informació molt curts cal una tassa de transmissió el més alta possible. D'aquesta manera s'aconsegueix que les radio tant del transmissor com la dels receptors estiguin el mínim temps possible actives.

Pel que fa a la potencia de transmissió es pot controlar de tal manera que si és necessari i no limita la comunicació es pot disminuir per gastar menys energia. Tot i així, en el paquets que s'envien es menciona la potència amb la que s'ha transmès per a que el receptor pugui estimar la distància fins al transmissor(?).

La modulació utilitzada per BLE és la mateixa que pel IEEE 802.11, GFSK (\textit{Gaussian Frequency Shift Keying}). Aquesta modulació és una de les més robustes, simples d'implementar i és el que permet (en part) a BLE tenir un abast molt més gran que Bluetooth Clàssic. Aplicar el filtre gaussià és útil ja que permet reduir el consum pic de energia \cite{BLE_Review} i també redueix les interferències en freqüències veïnes.
En Bluetooth 4.0 s'utilitza una desviació freqüencial de 185 kHz, en BLE 5 com que augmenta la velocitat de símbols també creix la interferència intersimbòlica. Per mitigar aquest efecte negatiu la desviació de freqüència passa a ser de 370 kHz


\subsubsection{Link Layer}

La capa d'enllaç és la encarregada de escanejar, anunciar i gestionar connexions amb altres dispositius

Per mitigar l'efecte de les interferències el protocol utilitza salts en freqüència (\textit{frequency hopping}). Això permet reduir l'impacte d'una interferència estreta (?). Els salts són de entre 5 i 16 canals d'entre els 37 dedicats a dades.

Bluetooth també permet la implementació de salts en freqüència adaptatius on només s'utilitzen canals suficientment bons descartant aquells en que es considera que hi ha masses interferències.

\subsection{Host}
\subsubsection{L2CAP}
La \textit{Logical Link Control and Adaptation Protocol} és la capa encarregada de l'establiment de la connexió lògica, multiplexament de protocols, segmentació i 'reasembly', control de flux per cada canal L2CAP.

La multiplexació de protocols és necessària tal que serveix per identificar quin és el protocol que s'utilitza en les capes superiors.

Per limitació física de l'arquitectura existeix una MTU (\textit{Maximum Transmission Unit}) i per tant els paquets de les capes superiors es converteixen en paquets més petits per a les capes inferiors.
Aquesta MTU es pot definir per cada connexió així flexibilitzant la varietat de dispositius amb els que es pot connectar un mateix dispositiu.

QOS Aquesta capa també és la que fa el seguiment de la qualitat de la connexió i dels recursos utilitzats per assegurar-se que les necessitats dels serveis es compleixen.

\subsubsection{SMP}
La capa \textit{Security Management Protocol} proveeix de diferents serveis relacionats amb la seguretat de la connexió.
Aquests serveis són: autenticació i autorització de dispositius i també integritat, confidencialitat i privacitat de les dades.
El protocol té tipus d'emparellament i generació de claus flexible per aconseguir reduir els requeriments de memòria i energia.

\subsubsection{ATT \& GATT}
El \textit{Attribute Protocol} és el protocol d'aplicació més comú per a BLE i el \textit{Generic Attribute Profile} defineix com utilitzar el protocol per oferir serveis a capes superiors.
El ATT és un protocol dissenyat per a dispositius \textit{Low Energy} amb l'objectiu de minimitzar la quantitat de dades transmeses. El atribut està format per 4 elements, \textit{handle}, UUID, permisos  i \textit{value}.

El \textit{handle} fa la distinció única entre els diferents atributs, ocupa 16 bits i no és obligatori que siguin valors seqüencials. És molt útil ja que s'utilitza per referenciar el atribut amb el mínim de bits possibles.
El UUID (\textit{Universal Unique IDentifier}) identifica el tipus d'atribut, aquest numero pot ser de 16 bits si s'utilitza algun que ja estandarditzat pel SIG o bé en tindrà 128 bits si està definit pel fabricant.
En els permisos s'indicarà quin tipus d'accés té el client a la informació (només lectura, lectura i escriptura ...). També pot estar definit si requereix un nivell mínim d'encriptació o si es necessari l'autenticació.
Per últim el valor del atribut serà on hi ha la informació en si i la seva longitud i interpretació dependrà del UUID tot i que té un limit de 512 bytes.

Des del punt de vista del protocol els dispositius són clients i servidors, normalment el client pren la iniciativa demanant dades però el servidor també te la capacitat d'iniciar una comunicació per exemple notificant quan un valor ha canviat.

La definició del ATT és massa genèrica per si sola tal que seria comú que per fer el mateix es desenvolupessin múltiples definicions que fossin incompatibles entre sí.
Per tal de tenir millor definits els serveis s'utilitza el GATT. El GATT permet definir perfils que agrupen múltiples atributs en un sol servei \cite{services}.

\begin{figure}[h!]
	\begin{center}
		\includegraphics[width=0.3\textwidth]{./images/GATT_Hierarchy.png}
		\caption{Jerarquia de GATT \cite{GATT_Hierarchy}}
	\end{center}
\end{figure}

En un llistat de atributs el GATT identifica els serveis tenint en compte que cada servei comença amb un atribut amb el UUID 0x2800, els següents atributs formaran part del mateix servei fins que es trobi un nou atribut amb UUID 0x2800. En el valor del atribut on es defineix el servei (UUID 0x2800) hi ha un altre UUID que especifica quin tipus de servei és.

Cada servei té característiques \cite{characteristics} que s'identifiquen perquè el UUID és 0x2803, en aquests atributs en el seu valor hi haurà un nou UUID que identifica la informació que es troba en el atribut amb un cert \textit{handle} que també estarà indicat.

\begin{center}
	\begin{tabular}{|l|l|l|l|}
		\hline
		Handle	&	UUID	&	Descripció						&	Valor		\\ 	\hline
		0x0100	&	0x2800	&	Battery Service					&	UUID 0x180F	\\		\hline
		0x0101	&	0x2803	&	Characteristic: Battery Level	&	\parbox[t]{4cm}{UUID 0x2A19	\\ Value handle: 0x0102}	\\	\hline
		0x0102	&	0x2A2B	&	Battery Value					&	20	\\	\hline
		0x0103	&	0x2800	&	Custom Temperature Service		&	UUID 	706676c8-3e49...	\\	\hline
		0x0104	&	0x2803	&	Characteristic: Temperature		&	\parbox[t]{4cm}{UUID 0x2A6E	\\ Value handle: 0x0105}	\\		\hline
		0x0105	&	0x2A6E	&	Temperature Value				&	25.45	\\	\hline
		0x0106	&	0x2803	&	Characteristic: date/time		&	\parbox[t]{4cm}{UUID 0x2A08	\\ Value handle: 0x0107}	\\		\hline
		0x0107	&	0x2A08	&	Date/Time						&	1/1/1980 12:00	\\
		\hline
	\end{tabular}

Exemple de possibles atributs
\end{center}

En aquest exemple es pot veure que tenim 2 serveis diferents ja que hi ha 2 UUIDs 0x2800 i tenim 3 característiques en total (una pel primer servei i dues pel segon) ja que hi ha 3 atributs amb UUID 0x2803.
Com que els calors corresponents al nivell de bateria i la temperatura estan estandarditzats no cal especificar a que es refereixen a percentatge de bateria restant i a graus Celsius.

Es pot veure com el servei que està definit per a la temperatura no forma part de l'estàndard ja que té 128 bits. El número que s'ha escollit és un UUID completament aleatori: 706676c8-3e49-4ecc-9379-fa9851444e53. Tot i que no hi ha una coordinació per assegurar-se que diferents desenvolupadors no utilitzen el mateix UUID degut a la longitud (128 bits) es considera improbable. La condició que ha de complir el UUID és que no sigui XXXXXXXX-0000-1000-8000-00805F9B34FB ja que aquest sufix correspon als que estan reservats per a l'estandard. En cas de voler tenir un UUID global reservat es pot fer pagant \$2.500 i també es poden veure tots els que ja s'han reservat per a empreses a questa web \cite{reservedUUIDs}.

En aquest exemple no n'hi ha cap però les característiques poden tenir descriptors \cite{descriptors} que permeten aportar informació addicional sobre la característica que els precedeix.

\subsubsection{GAP}
En quant es realitza una connexió els dispositius han de definir-se entre els següents rols: Anunciador (\textit{Advertiser}) o Escàner (\textit{Scanner}), Esclau (\textit{Slave}) o Mestre (\textit{Master}) i Emissor (\textit{Broadcaster}) o Observador (\textit{Observer}).
Aquests rols són independents per cada connexió per tant un dispositiu pot ser mestre en una connexió i alhora esclau en una altre.

Per iniciar una connexió entre dos dispositius (\textit{Peer-to-Peer}) un dispositiu vol ser descobert i envia missatges anunciant-se. L'altre dispositiu, que es vol connectar passa a ser un escàner. Aquest envia un paquet amb el requeriment de connexió, un cop acceptat, el dispositiu que fa d'escàner passa a ser mestre i el que anunciava passa a ser esclau.

També és possible transmetre informació des d'un dispositiu a tots aquells que estiguin escoltant, per tant, una comunicació de un a molts (\textit{One-to-many}). En aquest cas aquell que vol transmetre informació és el emissor i els dispositius que escolten són observadors.

\begin{figure}[h]
	\begin{center}
		\includegraphics[width=1\textwidth]{./images/rols_unicast.png}
		\caption{Pila de BLE \cite{ble_stack}}
		\label{ble_stack}
	\end{center}
\end{figure}

