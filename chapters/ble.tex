\chapter{Origens del Bluetooth Low Energy}\label{C:compaginacio}

%\section{Ús lliure de RF}
%Amb l'establiment de les bandes ISM (Afegir referencia) es crea la possibilitat per el públic general utilitzar comunicació inal·làmbrica amb molta facilitat. Això es deu a que l'establiment de les bandes ISM està acceptat de manera (mes o menys(?)) de la mateixa manera internacionalment.
%Això facilita el disseny de productes per al consumidor habitual que ràpidament veu els avantatges de la comunicació inal·làmbrica entre dispositius.
%En aquest entorn sorgeix la necessitat d'establir estandards entre le companyies principals dels sectors. Per definir els estandards s'agrupen les companyies i formen grups com al WIFI Alliance o ...

%\subsection{Wifi}
%La tecnologia de comunicació més popular avui en dia és la WIFI, dissenyada originalment als anys (?) i establerta als anys (?) orientada a permetre una connexió (* bona) i sense consideració per les interferències entre si mateixa que no es podien preveure llavors.  
%- Avantatges
%- Innovació
%- Evolució

\section{Història de Bluetooth Clàssic}
El desenvolupament de tecnologia per a connexions de curt abast que acabarien esdevenint el que avui es coneix com a Bluetooth Clàssic va començar l'any 1989 per part de Ericsson.
Inicialment la voluntat era utilitzar aquesta tecnologia per a connectar auriculars inal·làmbrics.
A IBM es volia integrar connectivitat de la xarxa de telefonia als ordinadors portàtils però no era factible degut al consum d'energia que requeria la tecnologia mòbil.
IBM i Ericsson van acordar utilitzar aquesta tecnologia de curt abast en els seus productes corresponents.
El resultat va ser que els mòbils Ericsson i ordinadors ThinkPad es podien comunicar entre si i així des del portàtil es podien fer trucades.
Com que ni Ericsson ni IBM tenien majoria en la quota de mercat dels respectius productes van decidir que la tecnologia fos oberta.
D'aquesta manera es buscava integrar a més participants en aquesta tecnologia per tal de estendre-la a la majoria de dispositius possible.
El 1998 es van unir al grup Intel, Nokia i Toshiba i totes 5 companyies  van fundar el SIG  (\textit{Blueooth Special Interest group}).
Finalment, el 2001 van sortir a la venta el primer mòbil amb Bluetooth, el Ericsson T39 i el primer portàtil el IBM ThinkPad A30.

\section{Història de Bluetooth Low Energy}
Com que el món de les comunicacions estava anant cap als dispositius sense fils i alimentats amb bateria era necessari adaptar les tecnologies existents per a les noves necessitats.
El 2001 a Nokia es va començar a desenvolupar una versió de Bluetooth que fos similar però que reduís significativament el consum d'energia amb el mínim de compromisos possibles.
El 2004 es va publicar la \textit{Bluetooth Low End Extension} \cite{Original_BLE_Extension}. 
Després de més desenvolupament juntament amb Logitech i MIMOSA [citation needed] es va divulgar al 2006 amb el nom de Wibree.

\begin{figure}
	\begin{center}
		\includegraphics[width=0.6\textwidth]{./images/Wibree_Logo.png}
		\caption{Logo de Wibree }
	\end{center}
\end{figure}

Els membres del SIG després de negociar entre sí, van acordar incloure Wibree al estandard de Bluetooth en la especificació 4.0 amb el nom de \textit{Bluetooth ultra low power technology} i publicitat com a Bluetooth Smart. El primer mòbil a incloure'l va ser el iPhone 4S al 2011.
Posteriorment es va canviar el nom per \textit{Bluetooth Low Energy} (BLE d'ara endavant)

\section{Bluetooth vs BLE}
El Bluetooth Low Energy no té cap relació amb el Bluetooth Classic pel que fa a l'arquitectura de la tecnologia.
Tot i que comparteixen l'ús de la banda freqüencial de 2,4 GHz, igual que altres protocols sense fils, i contenen el nom de Bluetooth queda clar que no estem parlant del mateix amb el simple fet que no son compatibles entre sí.
De fet, quant un dispositiu vol implementar les dues tecnologies (anomenat mode dual) ho ha de fer per separat ja que només comparteixen l'antena, les modulacions i altres blocs són tots diferents.

\section{MANETS (\textit{Mobile Ad-Hoc Networks})}
Bluetooth Clàssic no permet tenir més d'una connexió establerta amb dispositius.
A demés tot i voler transmetre poques dades l'arquitectura fa que es consumeixi bastanta energia per establir i mantenir la connexió.
Això no suposaven problemes inicialment, ja que principalment l'ús era per la transferència de fitxers com contactes o fotografies.

Amb l'arribada dels telèfons intel·ligents, els auriculars inal·làmbrics van esdevenir populars i es va establir la dominància de Bluetooth Clàssic per a escoltar música.
Per a poder transmetre música d'alta fidelitat es va millorar Bluetooth per ser més resistent a interferències i aconseguir velocitats superiors.
Per un altre costat però va començar a sorgir el Internet of Things, basat en tenir molts dispositius connectats transmetent a taxes molt variades i de forma discontinua.
Això va generar necessitats que no es podien cobrir amb les tecnologies existents.
Era necessari tenir xarxes sense fils, amb consum molt baix d'energia i que cobrissin una gran distància (10-100 metres).

Amb aquests requeriments, van sorgir les \textit{Mobile Ad-Hoc Networks} (MANETS d'ara endavant). BLE és una d'elles, però n'hi ha d'altres similars que s'analitzaran a continuació.

\subsection{Altres MANETS}

\subsubsection{Zigbee}
El estàndard Zigbee està orientat al control i seguiment de dispositius alimentats amb bateria. Zigbee està disenyat per sobre del IEEE 802.15.4 i aconsegueix una taxa màxima  de 250 Kbps i fins a 100 metres en línia de visió (\textit{LOS}).
Els dispositius de Philips Hue que serveixen per controlar bombetes intel·ligents entre d'altres.
[citation needed]

\subsubsection{6LoWPAN}
IPv6 \textit{over Low power Wireless Personal Area Networks} és un protocol que permet enviar paquets de Internet Protocol versió 6 a través de IEEE 802.15.4.
Aquest protocol està orientat a aportar internet amb connexions sense fils.
Una de les característiques destacades és la capacitat de comprimir les capçaleres dels paquets per així reduir el sobrecost que suposa.
La utilització de aquest protocol és majoritàriament coneguda per l'estandardització de Thread que utilitza el 6LoWPAN per a domòtica. [citation needed]


\subsubsection{Z-Wave}
Z-Wave és un protocol utilitzat principalment per domòtica. La comunicació es fa en una xarxa en malla que connecta els electrodomèstics i per tant permet el seu control.
La principal diferència d'aquest protocol es que utilitza les freqüències ISM de la banda 800/900 MHz (segons continent) i així evita les interferències que hi ha a 2.4 GHz de WIFI o Bluetooth.
La utilització de freqüències més baixes permet més abast amb menys potència, especialment quant es tracta de penetrar parets.
Un exemple de dispositius que utilitzen aquesta tecnologia són els productes de la marca Ring principalment coneguts pels seus vídeo-porters intel·ligens.


\subsubsection{Insteon}
Insteon està orientat a domòtica i utilitza conjuntament radiofreqüència i les línies d'alimentació de casa per a la comunicació.
Aquest sistema s'anomena de malla dual.
El fet d'estar connectats tots els dispositius a la instal·lació elèctrica permet molt bona sincronització entre els dispositius.
Això permet, per exemple, que múltiples dispositius puguin retransmetre paquets alhora per tal de millorar la cobertura i reduir les retransmissions.
SmartLabs és la companyia que fabrica i ven els productes que utilitzen Insteon.
La tecnologia és pot controlar a través de Cortana, Alexa o Apple Watch entre d'altres.

\subsubsection{LoRa}


\subsection{Comparació}
Tot i que els protocols tenen molt en comú cada un es diferencia de la resta en els detalls, a continuació veiem una comparació de les capacitats de cada un dels protocols


\section{Versions de BLE}
En quant al estàndard Bluetooth Low Energy cal diferenciar les característiques de la versió 4.0 amb la 5 ja que hi ha canvis significatius \cite{BLE_5_improvement_over_4}.

\subsubsection{LE 2M}
LE2M (\textit{Low Energy 2 Mega}) és el nom que te la nova capa física quan opera a 2 Msimbols/s en la versió 5 enlloc dels 1 Ms/s que es podia anteriorment en la versió 4.0.
L'augment de la velocitat dels símbols permet augmentar la quantitat de dades que es poden transmetre a nivell d'aplicació.
Aquest canvi està pensat per facilitar la utilització de BLE en situacions més exigents pel que fa a volum de dades com per exemple múltiples mesures del cos humà.

\subsubsection{Abast}
Tot i que la tecnologia BLE ja de per si en la seva versió original (4.0) tenia una distància màxima de transmissió teòrica molt gran (100 metres sense obstacles), no era suficient per a tots els casos.
Per exemple, era necessari tenir més abast en els sectors com el de les cases intel·ligents.
El límit de l'abast ve definit per la potència màxima de transmissió (sigui per protocol o per normativa) i per el percentatge de bits que es reben incorrectament BER (\textit{Bit Error Ratio}) que es pot acceptar.
Bluetooth defineix la BER mínima en 0.1\% per tant quant un de cada mil bits és erroni es considera que la connexió és massa dolenta i que ja s'ha arribat al límit de distància.

Per tractar els errors en transmissió hi ha dues estratègies: detecció d'errors i correcció d'errors.
En la versió 4.0 de Bluetooth només s'utilitzava la detecció d'errors.
En Bluetooth 5 s'implementa correcció d'errors d'aquesta manera sense augmentar la potència transmesa es pot incrementar el abast màxim que pot tenir la comunicació.
El desavantatge és que més bits dels que es transmeten es dediquen a corregir errors per tant queda menys espai per a dades de l'aplicació.


El esquema de correcció d'errors (\textit{Forward Error Correction}) es fa amb un codificador convolucional que pot generar a la sortida més bits dels que entren. El codificador pot funcionar amb dos esquemes diferents s'anomenen S=2 i S=8.
Amb els diferents esquemes s'utilitza un \textit{code rate} diferent que pot augmentar la capacitat de correcció i augmentar el abast però disminuir la quantitat de dades que es poden transmetre.

\begin{figure}[h!]
	\begin{center}
		\includegraphics[width=0.8\textwidth]{./images/LE_PHY.png}
		\caption{Comparació de diferents capes físiques}
	\end{center}
\end{figure}

\subsubsection{Advertisement Extensions}
En Bluetooth 4.0 els paquets d'anunci (\textit{Advertisement Packets}) tenen 6 octets de capçalera i com a molt 31 de dades, és a dir, com a molt es poden transmetre 31 Bytes de cop.
Normalment aquests paquets es transmeten pels 3 canals d'anuncis el 37, 38 i 39.

En Bluetooth 5 per poder emetre (\textit{broadcast}) més dades el que es fa és enviar un paquet d'anunci on només s'indica la capçalera i un punter cap al canal per on s'enviarà el paquet complet.
Posteriorment, pel canal per on s'ha indicat s'envia el paquet de dades i en cas de necessitar més dades s'encadenen paquets.
I d'aquesta manera les dades només es transmeten una vegada ja que abans, calia transmetre-les en els tres canals d'anunci.
Aquest procediment està detallat més endavant \ref{Advertising_Extension_PDU}

En la versió 4.0 sempre hi havia una component aleatòria que marcava en quin moment s'enviava el paquet, això és comú i serveix per evitar col·lisions periòdiques.
El problema que genera és que les ràdios han d'estar més temps escoltant fins a rebre el paquet per culpa d'aquesta aleatorietat.

En BLE 5 el GAP defineix un mode síncron que permet establir un procediment d'establiment de paquets anunci sincronitzats.
Existeix una nova capçalera (SyncInfo) on s'indica amb exactitud el interval i variació (per evitar col·lisions) dels paquets.    

\subsubsection{Slot Availability Masks}
La tecnologia LTE (telefonia) està utilitzant cada vegada més espai freqüencial i en preparació de que s'utilitzi en freqüencials pròximes a la banda de 2.4 GHz ISM s'ha desenvolupat un sistema per indicar disponibilitat en el temps.
Per evitar les possibles interferències que causin altres tecnologies es defineix la SAM (\textit{Slot Availability Mask}) que permet identificar aquelles ranures de temps hi ha disponibilitat així bloquejant aquells moments on hi hagui interferències i així evitar-les.

\subsubsection{Improved Frequency Hopping}
Els salts en freqüència que utilitza la versió 4.0 estan definit per 12 seqüències predeterminades.
En BLE 5 s'utilitza una seqüència pseudo-aleatòria per determinar quins canals s'utilitzen. El algorisme en aquest cas \textit{channel selection algorithm \#2} i és més efectiu a evitar interferències i esvaniments per propagació multi-camí.

\subsection{Altres Protocols}
Bluetooth Low Energy no és l'únic protocol orientat a la connexió de dispositius amb baix consum energètic. També existeixen els següents protocols:



\section{Pila BLE}
Un cop vistes les característiques generals del protocol cal entendre com funciona per dins.
En entrar en detall queda clar com el protocol s'ha dissenyat de forma molt flexible 

\begin{figure}[h!]
	\begin{center}
		\includegraphics[width=0.4\textwidth]{./images/BLE_Stack_simplified.png}
		\caption{Pila de BLE \cite{ble_stack} [refer]}
		\label{ble_stack}
	\end{center}
\end{figure}

La pila pròpia de Bluetooth Low Energy esta dividida en dues parts, el controladors i el \textit{host}. Aquestes dues parts són independents i utilitzen el protocol \textit{Host Controller Interface} (HCI d'ara endavant) per comunicar-se entre si.
Aquest protocol pot estar implementat amb qualsevol protocol de transport físic com USB o UART.
Les dues parts de la pila poden estar implementades en el mateix xip anomenat configuració única o en xips separats anomenat configuració dual.


\subsection{Controller}
El controlador compren la capa física i la capa d'enllaç.
\begin{figure}[h!]
	\begin{center}
		\includegraphics[width=0.6\textwidth]{./images/controller.png}
		\caption{Controller Stack}
	\end{center}
\end{figure}

\subsubsection{Capa Física}
La capa física és la que s'encarrega de la comunicació anal·lògica modulant i desmodulant les senyals.
Tal i com ja s'ha comentat abans treballa a la banda de 2.4 GHz en 40 canals diferents

\begin{figure}[h!]
	\begin{center}
		\includegraphics[width=0.8\textwidth]{./images/ble_channel_assignment.png}
		\caption{Canals BLE}
	\end{center}
\end{figure}

Els canals es classifiquen es divideixen en 37 de dades i 3 d'anunci (\textit{Advertisment}).
En els canals d'anunci es vol tenir més qualitat ja que la informació que s'hi transmet és més important.
És per això, que els canals d'anunci es troben en el buits que deixen els canals WiFi més comuns (1, 6 i 11).

Amb BLE l'objectiu és transmetre dades de forma eficient, i és busca poder desactivar la ràdio el major temps possible.
Quant la ràdio passa menys temps activa s'aconsegueix reduir el consum d'energia.
Per aconseguir-ho, el millor és una tassa de transmissió el més alta possible.

Pel que fa a la potencia de transmissió es pot controlar, disminuint-la per gastar menys energia.
Això serà possible sempre que l'entorn ho permeti i el receptor pugui rebre el senyal correctament.
El paquets que s'envien indiquen la potència amb que s'ha transmès per a que el receptor pugui estimar la distància fins al transmissor.
Conèixer aquesta distància és útil per a certes aplicacions.

La modulació utilitzada per BLE és la GFSK (\textit{Gaussian Frequency Shift Keying}), la mateixa que la majoria de MANETS que es basen en el IEEE 802.11.
Aquesta modulació és una de les més robustes, simples d'implementar i és el que permet, en part, a BLE tenir un abast molt més gran que Bluetooth Clàssic.
Aplicar el filtre gaussià és útil ja que permet reduir el consum pic de energia \cite{BLE_Review} i també redueix les interferències en freqüències veïnes.
En BLE 4.0 s'utilitza una desviació freqüencial de 185 kHz, en BLE 5 com que augmenta la velocitat de símbols també creix la interferència intersimbòlica.
Per mitigar aquest efecte negatiu la desviació de freqüència passa a ser de 370 kHz.


\subsubsection{Link Layer}

La capa d'enllaç és la encarregada de escanejar, anunciar i gestionar connexions amb altres dispositius

Per mitigar l'efecte de les interferències el protocol utilitza salts en freqüència (\textit{frequency hopping}).
Això permet reduir l'impacte d'una interferència estreta (?).
Els salts que es fan són des de 5 fins 16 canals per salt d'entre els dedicats a dades.

%Bluetooth també permet la implementació de salts en freqüència adaptatius on només s'utilitzen canals suficientment bons descartant aquells en que es considera que hi ha masses interferències.

\subsection{Host}
\begin{figure}[h!]
	\begin{center}
		\includegraphics[width=0.6\textwidth]{./images/host.png}
		\caption{Host Stack}
\end{center}
\end{figure}

\subsubsection{L2CAP (\textit{Logical Link Control and Adaptation Protocol})}
La \textit{Logical Link Control and Adaptation Protocol} és la capa encarregada de l'establiment de la connexió lògica, multiplexament de protocols, segmentació i 'reasembly', control de flux per canal.

La multiplexació de protocols és necessària per aconseguir que BLE sigui flexible.
Permet que des de capes superiors s'utilitzin els protocols que es vulguin i no només aquells determinats pel estàndard de BLE.

Per limitació física de l'arquitectura existeix una MTU\footnote{La MTU és la mida màxima que pot tenir un paquet} (\textit{Maximum Transmission Unit}) i per tant els paquets de les capes superiors es converteixen en paquets més petits per a les capes inferiors.
Aquesta MTU es pot definir per cada connexió així flexibilitzant la manera que s'utilitza el protocol per a cada cas.

La L2CAP també és la que fa el seguiment de la qualitat de la connexió i dels recursos utilitzats per assegurar-se que les necessitats dels serveis es compleixen.

\subsubsection{SMP}
La capa \textit{Security Management Protocol} proveeix de diferents serveis relacionats amb la seguretat de la connexió.
Aquests serveis són: autenticació i autorització de dispositius i també integritat, confidencialitat i privacitat de les dades.
El protocol té tipus d'emparellament i generació de claus flexible per aconseguir reduir els requeriments de memòria i energia.

\subsubsection{ATT \& GATT}
El \textit{Attribute Protocol} és el protocol d'aplicació més comú per a BLE i el \textit{Generic Attribute Profile} defineix com utilitzar el protocol per oferir serveis a capes superiors.
El ATT és un protocol dissenyat per a dispositius \textit{Low Energy} amb l'objectiu de minimitzar la quantitat de dades transmeses. El atribut està format per 4 elements, \textit{handle}, UUID, permisos  i \textit{value}.

El \textit{handle} fa la distinció única entre els diferents atributs, ocupa 16 bits i no és obligatori que siguin valors seqüencials. És molt útil ja que s'utilitza per referenciar el atribut amb el mínim de bits possibles.
El UUID (\textit{Universal Unique IDentifier}) identifica el tipus d'atribut, aquest numero pot ser de 16 bits si s'utilitza algun que ja estandarditzat pel SIG o bé en tindrà 128 bits si està definit pel fabricant.
En els permisos s'indicarà quin tipus d'accés té el client a la informació (només lectura, lectura i escriptura ...). També pot estar definit si requereix un nivell mínim d'encriptació o si es necessari l'autenticació.
Per últim el valor del atribut serà on hi ha la informació en si i la seva longitud i interpretació dependrà del UUID tot i que té un limit de 512 bytes.

Des del punt de vista del protocol els dispositius són clients i servidors, normalment el client pren la iniciativa demanant dades però el servidor també te la capacitat d'iniciar una comunicació per exemple notificant quan un valor ha canviat.

La definició del ATT és massa genèrica per si sola tal que seria comú que per fer el mateix es desenvolupessin múltiples definicions que fossin incompatibles entre sí.
Per tal de tenir millor definits els serveis s'utilitza el GATT. El GATT permet definir perfils que agrupen múltiples atributs en un sol servei \cite{services}.

\begin{figure}[h!]
	\begin{center}
		\includegraphics[width=0.4\textwidth]{./images/GATT_Hierarchy.png}
		\caption{Jerarquia de GATT \cite{GATT_Hierarchy}}
	\end{center}
\end{figure}

En un llistat de atributs el GATT identifica els serveis tenint en compte que cada servei comença amb un atribut amb el UUID 0x2800, els següents atributs formaran part del mateix servei fins que es trobi un nou atribut amb UUID 0x2800. En el valor del atribut on es defineix el servei (UUID 0x2800) hi ha un altre UUID que especifica quin tipus de servei és.

Cada servei té característiques \cite{characteristics} que s'identifiquen perquè el UUID és 0x2803, en aquests atributs en el seu valor hi haurà un nou UUID que identifica la informació que es troba en el atribut amb un cert \textit{handle} que també estarà indicat.

\begin{center}
	\begin{tabular}{|l|l|l|l|}
		\hline
		Handle	&	UUID	&	Descripció						&	Valor		\\ 	\hline
		0x0100	&	0x2800	&	Battery Service					&	UUID 0x180F	\\		\hline
		0x0101	&	0x2803	&	Characteristic: Battery Level	&	\parbox[t]{4cm}{UUID 0x2A19	\\ Value handle: 0x0102}	\\	\hline
		0x0102	&	0x2A2B	&	Battery Value					&	20	\\	\hline
		0x0103	&	0x2800	&	Custom Temperature Service		&	UUID 	706676c8-3e49...	\\	\hline
		0x0104	&	0x2803	&	Characteristic: Temperature		&	\parbox[t]{4cm}{UUID 0x2A6E	\\ Value handle: 0x0105}	\\		\hline
		0x0105	&	0x2A6E	&	Temperature Value				&	25.45	\\	\hline
		0x0106	&	0x2803	&	Characteristic: date/time		&	\parbox[t]{4cm}{UUID 0x2A08	\\ Value handle: 0x0107}	\\		\hline
		0x0107	&	0x2A08	&	Date/Time						&	1/1/1980 12:00	\\
		\hline
	\end{tabular}

Exemple de possibles atributs
\end{center}

En aquest exemple es pot veure que tenim 2 serveis diferents ja que hi ha 2 UUIDs 0x2800 i tenim 3 característiques en total (una pel primer servei i dues pel segon) ja que hi ha 3 atributs amb UUID 0x2803.
Com que els calors corresponents al nivell de bateria i la temperatura estan estandarditzats no cal especificar a que es refereixen a percentatge de bateria restant i a graus Celsius.

Es pot veure com el servei que està definit per a la temperatura no forma part de l'estàndard ja que té 128 bits. El número que s'ha escollit és un UUID completament aleatori: 706676c8-3e49-4ecc-9379-fa9851444e53. Tot i que no hi ha una coordinació per assegurar-se que diferents desenvolupadors no utilitzen el mateix UUID degut a la longitud (128 bits) es considera improbable. La condició que ha de complir el UUID és que no sigui XXXXXXXX-0000-1000-8000-00805F9B34FB ja que aquest sufix correspon als que estan reservats per a l'estandard. En cas de voler tenir un UUID global reservat es pot fer pagant \$2.500 i també es poden veure tots els que ja s'han reservat per a empreses a questa web \cite{reservedUUIDs}.

En aquest exemple no n'hi ha cap però les característiques poden tenir descriptors \cite{descriptors} que permeten aportar informació addicional sobre la característica que els precedeix.
Els atributs també tenen propietats d'accés que defineixen quines accions es poden prendre. Es comentaran en un exemple real més endavant [link dinàmic].

\subsubsection{GAP}
En quant es realitza una connexió els dispositius han de definir-se entre els següents rols: Anunciador (\textit{Advertiser}) o Escàner (\textit{Scanner}), Esclau (\textit{Slave}) o Mestre (\textit{Master}) i Emissor (\textit{Broadcaster}) o Observador (\textit{Observer}).
Aquests rols són independents per cada connexió per tant un dispositiu pot ser mestre en una connexió i alhora esclau en una altre.

Per iniciar una connexió entre dos dispositius (\textit{Peer-to-Peer}) un dispositiu vol ser descobert i envia missatges anunciant-se. L'altre dispositiu, que es vol connectar passa a ser un escàner. Aquest envia un paquet amb el requeriment de connexió, un cop acceptat, el dispositiu que fa d'escàner passa a ser mestre i el que anunciava passa a ser esclau.

També és possible transmetre informació des d'un dispositiu a tots aquells que estiguin escoltant, per tant, una comunicació de un a molts (\textit{One-to-many}). En aquest cas aquell que vol transmetre informació és el emissor i els dispositius que escolten són observadors.

\begin{figure}[h]
	\begin{center}
		\includegraphics[width=1\textwidth]{./images/rols_unicast.png}
		\caption{Establiment de connexió}
	\end{center}
\end{figure}


\section{Anunciaments}
Quant els dispositius volen transmetre informació o volen connectar-se entre sí el primer que cal fer és anunciar-se.
BLE és molt flexible alhora de configurar els paràmetres d'anunci i permet al desenvolupador adaptar el protocol per a les necessitats que tingui.
A continuació es concretarà quins son aquests paràmetres i com afecten a les prestacions finals del sistema.

Els paquets d'anunci com ja s'ha explicat anteriorment son aquells que serveixen a un dispositiu per donar-se a conèixer i alhora opcionalment transmetre informació \cite{Advertising}.

\subsection{Format}

\subsection{Tipus}
Hi ha 4 paquets d'aquest tipus i BLE 5 n'afegeix 4 més.
Els 4 originals es classifiquen segons si permet connexió i si permeten escaneig.

\begin{center}
	\begin{tabular}{|c|c|l|}
		\hline
		Connexió	&	Escaneig	&	Nom	\\	\hline
		Si			&	Si			&	ADV\_IND	\\	\hline
		Si			&	No			&	ADV\_DIRECT\_IND	\\	\hline
		No			&	No			&	ADV\_NONCONN\_IND	\\	\hline
		No			&	Si			&	ADV\_SCAN\_IND	\\	\hline
	\end{tabular}
\end{center}

El ADV\_IND és el genèric, més comú i permet que qualsevol dispositiu pugui connectar-se.
El ADV\_DIRECT\_IND serveix per a avisar un dispositiu específic, per exemple, si un rellotge intel·ligent que es vol connectar a un telèfon.
El ADV\_NONCONN\_IND només indica que existeix i no rebrà informació.
Això és útil per a balises, per exemple, per permetre localització en espais interiors.
El ADV\_SCAN\_IND també està orientat a balises però estarà escoltant per si rep missatges d'escaneig amb els que pot respondre amb poca informació.
Això permet una comunicació bidireccional limitada sense necessitat d'establir connexió.

Tots els paquets excepte el ADV\_DIRECT\_IND permeten transmetre 31 bytes de dades pròpies (han de seguir el format establert (?)).

\label{Advertising_Extension_PDU}En BLE 5 s'afegeixen 4 paquets més que permeten augmentar la quantitat de dades que es poden transmetre abans d'haver establert una connexió.
El ADV\_EXT\_IND és el paquet que s'envia pels canals d'anunci i indica en quin canal secundari s'enviarà el anunci. Aquest paquet no permet transmetre dades.
Els paquets que s'envien per canals secundaris tenen el prefix "AUX" i permeten enviar fins a 254 bytes de dades pròpies .
El AUX\_ADV\_IND és el paquet que s'envia per un canal secundari després que s'hagui indicat pel tipus de paquet anterior.
En aquest paquet es pot configurar si es permet o no connexió i escaneig però no les dues.

El AUX\_SYNC\_IND s'utilitza per indicar que s'enviaran anuncis periòdics pels canals secundaris.
D'aquesta manera no cal utilitzar els canals d'anunci tant sovint.
Per últim, el AUX\_CHAIN\_IND permet encadenar múltiples paquets (indicant quin serà el següent) per tal de transmetre encara més informació sense necessitat d'establir connexió.

\subsection{Paràmetres}
BLE permet seleccionar qualsevol combinació de canals per on transmetre els anuncis.
Per consumir menys energia es pot transmetre en un únic canal però això es recomana no fer-ho ja que si el canal és sorollós el dispositius no es podrà detectar.
Tot i la flexibilitat

Es pot configurar per quins canals d'anunci es vol transmetre els anuncis.
D'aquesta 
Interval d'anunci i canal

\section{Escaneig}

