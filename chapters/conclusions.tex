\cleardoublepage
\phantomsection
\chapter*{Conclusions}

Aquest projecte serveix per considerar si BLE és la tecnologia adequada per a les aplicacions de mesures mediambientals o biològiques.

\section{Tassa de dades necessària}

La tassa de dades que proporciona BLE és més que suficient per a l'ús que es considera en aquest cas.
La situació on la tassa de dades seria major és en mesures biològiques en temps real.

\section{Abast}

Pel que fa a la mesures biològiques no hi haurà problemes d'abast ja que el terminal i els sensors estaran sempre a menys de 10 metres.
En canvi per a mesures ambientals si considerem instal·lacions privades en habitatges, BLE té l'abast suficient fins i tot tenint en compte la penetració de parets per interiors.

\section{Energia Consumida}

BLE és dels protocols que permet consumir menys energia per a transmetre informació.
Això només serà un avantatge quan els sensors estiguin alimentats per bateria i la transmissió sigui el que consumeix més energia de tot el sistema.
Si els sensors tenen pantalles, s'alimenten de la instal·lació elèctrica, fan un processament de les dades dels sensors o emmagatzemen les dades localment, és possible que el consum de la transmissió no sigui significant.
En aquests casos no s'ha d'escollir el protocol de transmissió segons el consum ja que el temps entre recàrregues no es veu afectat pel protocol.


\section{Penetració del mercat}

És evident que la presencia de BLE en tots els mòbils interl·ligents facilita l'ús i simplifica la arquitectura d'un desplegament.
En cas de no utilitzar BLE no hi ha cap tecnologia de baix consum que es pugui connectar directament amb els telèfons dels usuaris.
Per tant, és necessari un dispositiu entremig entre els sensors i els usuaris que s'anomena concentrador o \textit{hub}.
En arquitectures simples suposa un cost extra però si es fa un desplegament extens, el cost és marginal.

En el cas de connexió directa fins al terminal del usuari BLE és perfecte ja que es troba en tots els telèfons intel·ligents.
En total es preveu la producció de 7.500 milions de dispositius entre 2020 i 2024 amb un creixement del 26\% anual segons el SIG\cite{Bluetooth_Market_Update_2020}.


\section{Interferències}

El protocol BLE només utilitza la banda de 2.4 GHz que és la més saturada de totes.
La previsió és que en el futur encara s'utilitzi més i per tant hi hagui més interferències que limitin la capcitat i latència de BLE.
Altres protocols 