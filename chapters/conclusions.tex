\cleardoublepage
\phantomsection
\chapter*{Conclusions}

Aquest projecte serveix per considerar si BLE és la tecnologia adequada per a les aplicacions de mesures mediambientals o biològiques.

%\section*{Taxa de dades necessària}

%Un dels dubtes a l'inici del treball era si aquesta tecnologia tenia una capacitat de transmissió suficient per a xarxes de sensors.
%Tot i que el Bluetooth Low Energy té bastant menys capacitat per transmetre dades comparant amb tecnologies com wifi, les dades necessàries en aquest tipus de xarxes acostumen a ser molt reduïdes.
%La tecnologia BLE no és capaç de transmetre vídeo o àudio en temps real però és suficient per transmetre fluxos de dades de sensors amb mesures biològiques o mediambientals.

\section*{Abast}

Com s'ha comentat, l'abast de la tecnologia BLE ve limitat tant per la distància entre transmissor i receptor com per les interferències.
En mesures biològiques es pot considerar que el transmissor i el receptor estaran relativament a prop, per tant, no hi haurà problemes d'abast.
En canvi, per a mesures ambientals si considerem instal·lacions privades en habitatges, cal tenir en compte els obstacles que s'oposen al senyal.
Tot i això tal com s'ha comprovat amb l'experiment de cobertura en interiors, BLE es pot utilitzar perfectament per a desplegaments domèstics de sensors.
També cal considerar la interferència més que probable de les tecnologies amb les quals BLE comparteix espectre com wifi, altres dispositius intel·ligents o amb Bluetooth Clàssic mateix.
Per combatre aquests desavantatges BLE té múltiples estratègies.
Primerament, sempre és possible augmentar la potència de transmissió (fins al límit permès), però reduint el cicle de vida de la bateria.
També es poden utilitzar capes físiques que implementen més redundància (\textit{LE Coded}).
Finalment, BLE implementa salts en freqüència i la \textit{Slot Availability Mask} per combatre interferències.

Totes aquestes tècniques augmenten la complexitat del protocol però ajuden a poder tenir un gran abast.
A més a més, BLE utilitza canals de freqüència estrets comparats amb altres tecnologies com wifi per tant, les interferències afecten menys a BLE que a altres tecnologies amb canals més amples.

\section*{Consum d'Energia}

BLE és dels protocols que permet consumir menys energia per transmetre informació.
Això només serà un avantatge quan els sensors estiguin alimentats per bateria i la transmissió sigui el que consumeix més energia de tot el sistema.
Si els sensors tenen pantalles, s'alimenten de la instal·lació elèctricao fan un processament de les dades és possible que el consum de la transmissió no sigui significant.
En aquests casos no s'ha d'escollir el protocol de transmissió segons el consum, ja que el temps entre recàrregues no es veu especialment afectat pel protocol.

En els escenaris on és important tenir un consum baix d'energia, el protocol BLE és una molt bona opció.
Com ja s'ha comentat múltiples vegades, BLE defineix moltes opcions d'implementació i això suposa una flexibilitat molt important.
Aquesta flexibilitat permet, per exemple, poder prescindir de les transmissions redundants com el reconeixement.
Tanmateix, es pot utilitzar la latència d'esclau per reduir els esdeveniments d'una connexió quan no són necessaris.
També, permet transmetre molta informació sense necessitat d'establir una connexió.
Així doncs, poder evitar realitzar tots aquests procediments permet reduir significativament el consum que suposa la transmissió de dades i fa possible augmentar el cicle de vida de la bateria o reduir la capacitat d'aquesta.
Gràcies a BLE és possible que dispositius no hagin d'estar connectats a cap altre dispositiu a través de cables ni a la xarxa elèctrica.

\section*{Penetració del mercat}

Quan en un projecte hi ha usuaris involucrats d'alguna manera, és molt probable que el telèfon mòbil sigui un component important.
BLE té el gran avantatge que ve incorporat en tots els mòbils intel·ligents.
Això facilita l'ús, ja que els usuaris estan més acostumats, i simplifica l'arquitectura d'un desplegament.
L'arquitectura és més simple perquè el mateix dispositiu mòbil és el receptor de les dades.
En cas de no utilitzar BLE, no hi ha cap tecnologia de baix consum que es pugui connectar directament amb els telèfons dels usuaris.
Altres tecnologies acostumen a utilitzar un dispositiu entre els sensors i els usuaris que s'anomena concentrador o \textit{hub}.
A l'afegir un nou dispositiu s'incrementa el cost d'un desplegament.
A més a més, augmenta la complexitat del flux de dades i també incrementa l'ús de la banda freqüencial, ja que calen més transmissions per la mateixa quantitat d'informació.

En canvi, en el cas de connexió directa fins al terminal de l'usuari BLE és perfecte.
No només està present en telèfons mòbils sinó també en: joguines, televisors, electrodomèstics, càmeres, comandaments a distància, entre d'altres...
En total es preveu la producció de 7.500 milions de dispositius entre 2020 i 2024 amb un creixement del 26\% anual segons el SIG \cite{Bluetooth_Market_Update_2020}.